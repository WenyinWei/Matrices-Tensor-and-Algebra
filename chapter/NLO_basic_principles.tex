

\section{非线性光学的基本原理}
本章将以绝缘介质为对象,建立非线性光学的基本原理。这个原理包括非线性极化和极化率的理论和耦合波方程。它们是理解非线性光学的基础。
\footnote{本章节内容几乎全部从高健存老师于清华大学 2019 年秋季学期开设的非线性光学课程讲义中整理,引用请注高健存老师的名。}



\subsection{非线性极化和非线性极化率}
介质中输入光波,会导致介质极化。极化强度是光波场强的函数,在弱光下,极化强度是场强的线性函数,在强光下,它与场强成复杂的关系。这时就出现非线性光学效应。如果光场还不算太强,则极化强度可按总电场强度的幂级数展开

                    (1)                         

式中右边第一项与场强成正比是线性极化,  分别与场强的二次,三次乘积成正比,称为二阶,三阶等非线性极化强度。后面些项统称为非线性极化,所以(1)式又可写成

                                (2)

通常情况下非线性极化是对线性极化的微弱修正。虽然非线性极化是产生非线性光学效应的主要项,但是在观察非线性效应时,线性极化有重要影响。

非线性光学中常用激光作光源。当激光的线宽远小于光频本身时, 可将它视为平面单色光。为了正确表示极化和场强的关系,首先将平面单色光场用复数形式表示,即将它表示为正频和负频两部分之和

           (3)

这里电场正频分量是 ,负频部分则是它的复共轭并以c.c.表示。其中 是正频分量的空间部分(即除去时间因子的部分),它包括复数振幅 ,和空间相位因子(即沿 方向的传播因子) ,  是初位相。        

                                  (4)

规定写        ,则(3)式可写成

                                     (5)

若同时有几个光波存在,其频率为 ,则总场仍以上式表示,但这时 
由于激光常近似为平面单色波,其频率只取分立值,与此对应,介质的极化强度也应是一些分立频率成分的总和,可以写成

                                             (6)

对比(1)式可知,

                 (7)

式中 是极化强度可能取的频率。由于极化强度是实数,所以 也取正频和负频。但是对不同阶的极化强度其可能取的频率是有不同的。
      对于分立频率情形,在频域中讨论常常比在时域中讨论更方便。
对于线性极化,在频域中它的每个频率分量必定与电场的同频率分量成线性关系,因此线性极化所有可能的频率只能是光波的频率。 并有关系

                          (8)                 

式中引入比例系数 ,称为线性极化率。它是反映物质对光场响应程度的系数,与物质结构有关。对不同频率的光,介质的响应不同,所以线性极化率是频率的函数,这一性质可称之为介质有色散。对各向异性介质线性极化率是二阶张量,对各向同性介质线性极化率是标量。将(8)式代入线性极化式中,就有在时域中的表示式

                              (9)

首先我们注意到,由于介质的色散,线性极化与光场之间不存在时间的定域关系,即
            
但(9)式表示空间有定域关系。这在光学频段的多数情形中是成立的。以后我们不讨论极化与空间不同地点的场强有关联的情形,即空间非定域关系。

对于二阶非线性极化,按定义 中的所有频率成分应当和场强二并矢的各个频率成分相对应。由于存在色散,在频域中可以引入二阶非线性极化率张量用以建立以下关系

        (9) 

式中左边时间因子 应和右边的时间因子 对应相等,所以二阶非线性极化可能的频率应是光频所有的二次组合。这些不同频率的极化就是产生次级电磁波的波源。由(9)是可知,任一频率成分的二阶非线性极化其空间部分可表示为

                             (10)

式中已略去两边的时间因子。 为二阶非线性极化率,它是个三阶张量,为了表示色散,它与频率有关。由于光频中可能有几个不同的二次组合对应同一个极化频率,所以用 表示对特定频率 的所有可能的光频二次组合。我们假定空间关系是定域的所以空间坐标已经略去。(10)式的坐标投影形式为

                 (11) 

式中 表示直角分量,并采用爱因斯坦求和规则。

上述讨论可推广到三阶和更高阶非线性极化。例如对于三阶非线性极化我们可以将它的所有频率成分与光场的三并矢的各个频率成分建立联系,通过引入三阶非线性极化率张量就得到

                      (12)

比较此式两边可见,三阶非线性极化可能的频率应是光频所有的三次组合。 。任一频率成分的三阶非线性极化的空间部分可表示为

      (13)

式中 为三阶非线性极化率,是个四阶张量。当所有频率成分已决定,则时域中三阶非线性极化就是

                                               (14)

更高阶非线性极化和非线性极化率可以类似建立。

\subsection{非线性光学极化率的对称性}
线性和非线性光学极化率是介质的属性即它与介质的结构有关,也与介质在光的作用下的激发和弛豫等动力学有关。介质结构上的和动力学的对称性决定了光学极化率的对称性。由于介质的各阶极化率是一些张量,而一个张量有许多张量元素,极化率的对称性将表现在各种张量元素之间存在一定的关系,即它们并不是完全独立的。利用这些对称性将使得非线性光学的表达和运算得以简化。

首先,由于光场是实数则由§1.1(5)式可得到 ,同样各阶极化强度也应是实数,故有关系 。由此可得到各阶非线性极化率的关系

                           (1)
                           
此式表示负频部分的极化率等于其正频极化率的复共轭。这是极化率的一个普遍关系。根据这个关系,如果正频部分的极化率已知,则负频部分也就知道。

其次,一个n阶非线性极化率必定具有固有交换对称性。以二阶非线性极化为例,它可写成分量形式  

                        (2)

此式也可写成

                             (3)

上式中第二等式是将下标 而得,其结果不变。比较(2)(3)可见,对任意场强的值此二式要相等,必有

                            (4)

即将张量元素中的“数对” 交换其值不变。这就叫做二阶非线性极化率的固有交换对称性。这一结果很容易推广到更高阶极化率,即固有交换对称性可表示为

                             (5)

式中算符 代表“数对” 的任何交换。这一对称性与物质的具体性质无关,它仅仅表示介质对光场的响应和光场作用的次序无关。所以和(1)式一样,表示极化率固有交换对称性的(5)式也是极化率的普遍的关系。

利用固有交换对称性,可以简化非线性极化的表示式。设在频率为 的光波作用下写出和频为 的二阶非线性极化的分量(空间部分),由(2)式可得

       
而由于二阶极化率的固有交换对称性(4),右边二项的贡献是相等的,所以可以合并写成

用张量符号表示则为


但是要表示频率为 的光波作用下产生二次谐波,则频率为 的二阶非线性极化的分量(正频的空间部分)应是

可见其极化率前的数值因子与和频时不同。

综合以上情形,可以利用固有交换对称性将一般情形的二阶非线性极化统一写成

                        (6)

这里D为 的不同排列数。当 ,D=2,当 并且二束光不可区别时D=1,二束光可区别时,仍取D=2。但是当使用(6)式时,其中的所有频率不能有零频。零频的出现需修改(6)式分母上的数值因子。

利用固有交换对称性可将n阶极化写成

                     (7)

其中D是n 个频率的不同交换数。当n 个频率中有 个频率是 , 个频率是 等等,则 。 使用(7)式时 ,要注意式中也不应包括零频在内,零频的出现需修改(7)式分母上的数值因子。在 个频率 中若有几个是可区别的模,计算D时应将它们视为不同频率的模。此外如果有几组不同光波频率组合对应同一个极化频率,则须分别将它们写出来。

以下再介绍在有限制条件下非线性光学极化率的一些对称性。在非线性光学的研究中,许多情形是光束工作在介质的光学透明区,即光能量在介质中无损耗。在此条件下非线性光学极化率具有全交换对称性。它表示为

                    (8)

式中的算符 为任意数对的交换,包括分号前的数对 与其它数对的任何交换。
此外可以证明,对无损耗介质,不仅线性极化率是实数,而且各阶非线性极化率也是实数。这一性质又叫做非线性极化率的时间反演对称性。它可以表示为

                        (9)

利用这些有限条件下的对称性能使得非线性光学的处理得到简化,这一对称性常常能在相同阶的不同非线性光学效应的极化率分量之间建立关系。因此测定一种非线性光学效应的极化率分量就可能知道另一种同阶的非线性光学效应的极化率分量。例如三波混频中由 的和频过程的极化率分量就可得到差频过程 的极化率分量。(它们的非线性极化率的分量是一一对应的)。

特别对于弱色散的介质,还可近似得到一个很实用的对称性—克莱曼(Kleiman)对称性。克莱曼对称性是指如果将一非线性极化率的某个张量元素的下标进行交换,则这些元素的值不变。例如二阶非线性极化率的一个元素是 满足克莱曼对称性时便有

                      (10)

即只将极化率的下标任意交换(频率不相应交换),这些极化率分量的值不变。这一对称性使得同一个过程极化率的独立分量数目大为减少,这在应用中是很重要的。

除以上的对称性以外,由物质结构的对称性也给光学极化率加上限制。这叫做极化率的空间对称性。介质,特别是晶体,具有一定的宏观空间对称性,它将对表示介质特性的各种张量加上限制,因而使各阶张量的分量之间建立关系,并且使的独立张量元素的数目大为减少。

晶体的空间对称性是按照格胞的几何形状和各种对称操作元素来分类的,按照点群分类共有7个晶系,32个晶类,如图所示。
       
对称操作包括定轴转动c(有对称轴),镜面反射 (有对称面),中心反射i(有对称中心)和旋转倒反s四种。如果上述座标变换是按对称元素来操作的,则操作前后晶体的空间位置不会改变。

用空间对称性求出张量分量之间的关系叫做张量的约化。  设晶体有一个二阶张量 ,它可以代表线性极化率张量也可以是描写其它物性的量,经过座标变换,变成 ,由张量定义

                                                     (11)

式中 等是坐标的变换矩阵元。如果坐标的变换对应一个对称操作为 ,对此种操作便有

                    。                                            (12)

联合(11)(12)两式便可找到张量各分量之间的关系,从而减少极化率的独立分量数目。对于更高阶张量的约化可以类似进行。

以二阶非线性极化率(三阶张量)为例,它有27个元素,用矩阵表示可写成

          (13)

例如KDP晶体属于四方晶系, 晶类,它的对称操作元素有 。将它的二阶非线性极化率进行约化。

                   
利用全交换对称性可得到 ,其余元素为零。以矩阵表示则为
          
用空间对称性可以证明:具有中心反射对称性的介质不存在偶数阶的非线性光学效应。具有反射中心的介质包括原子分子气体,液体,非晶态的固体,以及某些具有反射中心的晶体等。

\subsection{古典非谐振子模型}
前面关于极化率的描述带有唯象的的性质。这种表述虽然在形式上是普遍的但不能得到有关各阶极化率的大小以及介质对光响应快慢的定量结果。要深入了解各种非线性光学的效应就必须分析介质的结构和极化的具体机制,建立各种极化的模型。我们首先介绍一维非谐振子模型.。当原子分子气体,分子晶体和有机高分子材料等介质和光没有共振相互作用时,这个古典的模型是一个初级的近似。

电子在原子分子内受一势场作用。
 
为简单起见,设振子是一维的,其势能可在平衡位置附近展开

                (1)          

这里U的一阶导数项为零, .。a, b 等是非谐系数。电子在势场中受力

                 (2)           

式中右边第一项是弹性力,如果忽略非弹性力的作用 则电子将在势场中以固有频率作谐振动,形成谐振子。当原子在光波作用下,电子还受电场E(t)的力。  所以电子的运动方程应为

                          (3)       

式中令振子振幅 ,  是振子的固有频率,m 是它的质量,  是振子的非谐系数。  表示对时间的一阶导数。考虑到原子不是孤立的,已加上阻尼力,Γ是阻尼系数。当光场足够强,它将驱动电子以较大的振幅振动,这时非谐力不可忽视。所以(3)式右边保留非谐力。上式可用微扰论求解。令

                                        (4)

代入方程(3),并令等式两边含相同阶微扰振幅的系数对应相等,则得

                         (5)

若令光场由平面单色波组成

                                            (6)

则振子的振幅也应是分立频率成分的组合

                    (7)

由(5)中第一个方程可解得振子单个频率成分的一阶微扰振幅

                          (8)          

式中定义                                          (9)

代入(7)式可得到振子一阶微扰的总振幅是

                                             (10)

实际上这一项是简谐振子在光场作用下的解。它的频率由光场频率决定.。将 代入(5)中第二个方程可解得振子单个频率成分的二阶微扰振幅

                (11)   
                                       
可见二阶微扰振幅的频率 由光场频率的二次组合决定。 表示对特定的ωσ所有不同的光频二次组合。此项与光场的二次乘积成比例,说明产生了非线性,并且是由原子的非谐系数 所导致的。将(11)代入(6)式可得到振子二阶微扰的总振幅是

                                   (12)     
将 代入(5)中第三个方程可求得 。它的频率是光频的三次组合,它的振幅与场强的三次乘积成正比,并由非谐系数 共同决定。由(9)(11)等式可见,振子振幅实际上是按场强的幂展开。

以上结果可用来求介质的极化强度。 由极化强度的定义 ,在忽略原子分子之间的相互作用时有  ,N是单位体积的原子数。 是振子的偶极矩。由于振幅已作微扰展开,所以极化强度也表成微扰展开的形式:

                                         (12)

前面唯象理论已经给出各阶的极化,在一维情形它们是
          
                   (13)

其中 。与古典振子模型结果对比便得到线性极化率和各阶非线性极化率的表示式。例如

                              (14)     

                                     (15)    

由线性极化率和各阶非线性极化率的表示式可见,它们的分母中都有一个或几个 因子。当光频或者其组合频率与原子的一对或数对能级的跃迁频率接近时,这些极化率会增大许多数量级。

在上述模型中,如果势能是中心对称的,由(1)式可知,右边奇数项的非谐系数如 等必为零,所以不存在偶数阶的非线性极化率。原子和许多具有中心反射对称性分子就是如此。所以它们的最低阶非线性极化是三阶。但是也有许多分子是各向异性的,它们不具有中心反射对称性,然而此类分子组成气体或液体时宏观上仍具有中心反射对称性,所以也不存在偶数阶的非线性光学效应。对这些介质要考虑 以上的振子振幅修正。 类似计算可得

在没有共振发生时由振子模型可以估算各阶极化率的大小。多数固体在紫外区或深紫外区有强吸收带,可以认为它们在光学区的的固有频率ω0≈1016 Hz,而可见光的频率则为 。所以线性极化率可近似表示成

                                                         (16)    

固体原子数密度N≈1022㎝-3. 代入电子电量和质量可得χ(1)≈1,相应于折射率 ,这一估算符合多数固体的实际情形。气体的粒子数密度比固体低许多数量级,所以其 ,它们的折射率近似为1。 

二阶非线性极化率, 可表示成

                                                 (17)     

这里须估计非谐系数α 的值。可以认为在光场作用下,当电子振幅偏离平衡值为一个玻尔半径时,由α项引起的非谐力接近弹性力。即  。由此得   ,a0是玻尔半径。将  代入(17)式得χ(2)≈10-12 m/V. 同样对非共振时的三阶非线性极化率

可计算得对固体χ(3)≈10-22 (m/V.)2。这些值也基本符合实际.(具体例子)

由古典振子模型还可得到极化按电场展开的收敛条件。 作比值
        
式中如果利用关系   ,这里 Eatom是原子内电场,则有

                                       (18)

此式表明,只有光场的电场强度小于原子内电场时,级数才收敛,即将极化强度展开为各阶非线性极化,并引入各阶非线性极化率才有意义。

当光与介质有共振时,微扰论的模型一般不适用,上述估计自然也不能用。 

\subsection{介质非线性极化的物理机制}
    
 上一节通过非谐振子模型讨论了原子分子介质产生非线性极化的起源。这是一种过于简化的模型,它忽视了物质的具体结构以及原子分子对光响应的多样性。实际上物质结构和状态的多样性决定了非线性极化机理的多样性, 揭示其机理正是非线性光学重要任务之一。 这里我们对常见的物理机制作一点定性的说明。

\begin{enumerate}
    \item 电子云的畸变.  
    \item 分子内部的核运动.  
    \item 布居数的变化.  
    \item 分子的取向.  
    \item 电致伸缩.  
    \item 热效应
\end{enumerate}

以上所列举的机制多数由介质的束缚电子对光场的响应而产生。其特点是极化与光强直接有关。还有其它的一些物理机制,例如光激发产生的自由电子或电子空穴对。这些载流子密度与光场的非线性关系导致由介质的响应的非线性。不过它们的极化不是直接与光强有关,具有非定域的性质。一般说它们的非线性比束缚电子的非线性要大,响应时间依受激发介质的不同而不同,有的极慢(如光折变材料)而有的极快(如半导体及其低维材料)。

\subsection{平面单色光在线性各向异性介质中的传播}

以上我们讨论了介质对光的响应。在平面单色波的近似下,建立了各阶非线性极化与场强的关系。介质极化也反作用于光场,影响光波的传播。光波在介质中传播由麦克斯韦电磁场方程决定。 对非铁磁性绝缘介质,由麦克斯韦电磁场方程可以得到光场的波动方程

                                (1)

上述波动方程的右边就是作为波源的极化电流,这个极化电流产生于随时间而变的极化强度。即 。这个波源会产生次级电磁波,这些次级电磁波与输入的光波(激发波)组合便可以描写介质中光的传播特性,产生包括线性和非线性的光学效应。

设作为波源的极化强度为几个不同频率的单色波之和

                                               (2)           

此时介质中的光波也是不同频率的单色波之和,

                                             (3)    

将(2)(3)代入波动方程, 其中频率为 的分量满足以下方程,

                            (4)

这一做法实际是作傅里叶变换,从时域换到频域,因为在频域中求解方程更加方便。在非线性光学中常常可以将极化强度分为线性极化和非线性极化两部分

                               (5)

这时光在介质中的传播特性主要由线性极化决定,非线性极化是对线性极化的微扰性修正。频域中线性极化与场强有线性关系

                               (6)

对于各向异性介质,线性极化率是个二阶张量,对各向同性介质,它是标量。可将(4)化成以线性介质为背景的波动方程。为此将(5)代入(4)并将线性极化这一部分移到等式左边,就得到

            (7)

式中线性相对介电张量为

                                              (8)

在各向异性介质中,它是二阶对称张量。当它是实数时, 在主坐标系中,它只有三个不为零的分量,即三个主值

                                             (9)

用矩阵表示则为

                                      (10)   

对于各向同性介质, 为标量。

                                   (11)

为了求解波动方程(7),我们首先研究光波在线性各向异性介质中的传播,即求齐次方程的解。

在波动方程(7)中,令 , 求平面单色波解。令  ,

代入方程得到

                    (12)                

令波矢为

                                                   (13)

 为传播方向的单位矢量, 为该传播方向的折射率. 代入(12)式,得

                                 (14)              
                                 
写成张量形式则为

                                           (15)

这是关于场振幅投影 的一组代数方程。若方程有非零解,其系数行列式必须为零,选主坐标系后就得到以下方程,通常叫菲涅耳(Fresnel)方程。它决定了不同传播方向折射率的大小。

            (16)

由式可见这是关于 的一个二次方程。对一特定的传播方向, 有两个独立解,对应两个不同大小的波矢和两个独立的本征模。

令 .代入(16)可得折射率面方程

           (17)      

在此空间一个径矢量的大小表示该方向的折射率: 。下面讨论几种介质
\begin{itemize}[(a)]
    \item 光学各向同性介质(立方晶系):三个主介电常数相等 
    
    此时(17)式变成     ,这是二个重叠的球面。它表示折射率与方向无关,同一方向两个本征模的电矢量彼此垂直并与波矢也垂直,折射率也相同。

    \item 单轴晶体:有两个主介电常数相等  , 叫主折射率。此时方程(17)成为
    
              (18)         
  
即有两个折射率面一个是球面,另一个是旋转椭球面。沿特定方向两个本征模叫做寻常光(o光)和非常光(e光),一般有两个不同的折射率。只是在一个特殊方向两个折射率才相同,这个方向叫光轴。按照主折射率的相对大小晶体分为正单轴晶体( )以及负单轴晶体 ( )。由(18)可求得传播方向与光轴成 角时两个本征模的折射率

              ,    与传播方向无关           (19)      

 即0光的折射率是 ,它不随传播方向而改变,e光的折射率则随传播方向而改变。 o光的电矢量与波矢垂直,e光的电矢量一般不和传播方向垂直。因此e光的能流密度的方向和波矢方向有一夹角,叫离散角δ,可以证明
                                 (20)    
    \item 双轴晶体:三个介电常数都不同,规定主轴的选择是 。   双轴晶体折射率面是个二层的复杂曲面。它存在两个光轴,它们在xz平面内与z轴对称的取向。沿任意传播方向有两个本征模,它们的折射率不同,因此称为快光和慢光,它们的电位移矢量彼此垂直并与波矢也垂直,但它们的电场强度一般与波矢不垂直。折射率面与三个主平面的交线如图(B)所示,每个主平面内的交线由一个圆和一个椭圆组成,在zx 面内圆和椭圆相交,产生两个光轴,光轴与z轴的夹角 满足
    
    图(A)是双轴晶体的折射率面,图(B)是折射率面与三个主平面的交线。
    
    通常将 的晶体叫做正双轴晶体,将 的晶体叫做负双轴晶体。
\end{itemize}

\subsection{定态耦合波方程    缓变振幅近似}
   当有光强足够大的激光作用下,介质的非线性不可忽略,此时前一节的波动方程(7)中  。我们讨论沿Z方向传播的平面单色波解。 为了简化计算,需要作一些近似。已知对各向同性介质,方程 成立。对各向异性介质,一般情形方程 不成立,但考虑到通常它们的介电张量三个主值近似相等,所以可以近似认为 也成立,同时介电张量也近似为标量。因此有

                                    (1)

沿z方向传播频率为 的平面单色光波可写成


极化波为  ,将它们代入前一节的波动方程(7)就可以得到

                   (2)       

进一步设光波的正频部分的解为

                                          (3)          

这里 为光波在线性介质中的波矢,其色散关系为

                                             (4)

 为其偏振方向单位矢量。这里与线性介质有不同的是已令振幅  为z的缓变函数。非线性的相互作用将使得光波和介质以及光波与光波之间必有能量的交换,故而一般说不能保持各波的振幅不变。将(2,3)代入波动方程并考虑到在光学频段,电磁波振幅在波长范围内变化很平缓 ,即取缓变振幅近似, 

                                   (5)     

上式就化成

                               (6)             

(6)是非线性光学中最基本的方程之一。它适用于均匀的各向同性介质也近似适用于各向异性的介质。

光与介质非线性相互作用通常有几个不同的光波参与其中,必须对每一个光场按照(6)式写出它的方程。由于每一方程中的源项 常与几个光场通过非线性极化率相关联,所以每一个方程不能单独求解,而要在适当的边界条件下将一组方程联立求解。因此(6)常称为耦合波方程。

此外须注意,由于光场的波动方程采用了缓变振幅近似,原来的二阶微分方程变成一阶微分方程,这时对沿 方向传播的光波必须分别列出它们的方程。对沿-Z方向传播的波,其耦合波方程中应以-k代替k。

如果考虑到激光的光束半径总是有限的而且在横截面上光强和相位分布是不均匀的,例如是高斯光束,特别是聚焦光束则更实际的近似是将激光看成非平面的旁轴的单色光。适合这类光的耦合波方程应加以适当修改。首先将非平面的旁轴单色光表示为

                            (7)

式中振幅已考虑到光场在(x,y)横截面上有分布。这时在(1)式中将拉普拉斯算子分为两部分

                              (8)

其中 是二维拉普拉斯算子 

再利用沿z方向的波的缓变振幅近似则耦合波方程就化为

               (9)

在线性介质中旁轴光的方程是

                                  (10)

可以证明它的解与菲涅尔-基尔霍夫衍射公式等价,就是说此方程可以描述有限光束的衍射效应。

对于光波导和光纤等有界介质,光的衍射消除但要代以特定的传播模式。其非线性光学效应也要用方程(9)。


\subsection{非定态耦合波方程}
实际的激光都有一定的线宽 。作为初级近似,可将激光看作单色光。但是当线宽的影响不可忽略时,以前有关单色光的公式和方程需要作一定的修改。特别是在非线性光学中,为提高光强常采用脉冲激光。脉冲光的谱线宽度Δω和它的脉冲宽度一般有关系 ·Δω~1。调Q脉冲的宽度约为ns量级,锁模脉冲的宽度约为ps至几十fs量级,光脉冲愈短则它的带宽愈宽,所以介质的色散不可忽略,并将影响光脉冲的传播特性。这时用定态耦合波方程不能正确描写光与介质相互作用的过程,而要使用非定态耦合波方程。然而在可见光频段,多数脉冲的线宽仍满足 ,这里 是谱的中心频率。我们将这样的光波称为准单色波。

要建立非定态耦合波方程.,我们首先从时域的波动方程(§1.5(1)式)出发,为简化表示,设介质是各向同性,光场取平面波。则波动方程是

                          (1)

将方程转到频域,并将线性极化与非线性极化分开,就得到

                           (2)

式中 是介质的线性介电常数。注意这里的频率是在线宽范围内的连续变量。 对于准单色光波,其线宽比中心频率要小很多,即 ,所以光场可以用载波以及包络来表示。设载波的频率是 ,则

                              (3)

它的正频成分的空间部分是

                        (4)      

当 时波包包络 是时间的慢变函数,也是空间的慢变函数。同样在准单色光波作用下介质的各阶极化强度也应该用准单色波表示

                            (5)

它的正频成分的空间部分是
     ,

是极化波的包络,它是时间与空间的慢变函数, 是载波频率, 是在此载波频率处极化波的波矢。由(3)式场强的傅氏分量(这里只写出正频部分,即 的频带。 负频部分做法类似)

                               (6)

  即                           (7)

  是场的包络在 附近的傅氏变换

                      (8)       

方程(2)右边非线性极化的傅氏分量同样可用其包络的傅氏分量表示。即其正频部分为

                               (9)

将上述准单色波的傅氏分量(7)和极化(9)代入(2)式,并取缓变振幅近似,注意对准单色波的包络不仅对空间的变化而且对时间的变化都是缓变的,即有

                                     (10)
                                     
这时就得到

                 (11)

式中右边 是非线性极化的包络的傅氏分量。 由于场的准单色性, 的频谱只在 附近很狭的频带内不为零。所以可将频率 处的波矢k在 附近展开

                              (12)

并利用 ,就得到

                 (13)

因此有

          (14)

利用付里叶变换的关系

          
将上式作逆付里叶变换,就得到时域中的非定态耦合波方程

                 (15)

式中系数        为波包的群速度。              (16)

第三项的系数

                       (17)

此系数反映群速度随频率的变化率,叫做群速度色散。

在(16)式中群速度决定波包峰的运行速度,它和波包的载波频率(或波长)有关,不同载波频率的波包群速度不同。群速度色散与波包的变形有关。一个超短脉冲的带宽很宽,包含许多频率成分,如果介质有色散,则不同的频率成分便有不同的速度,经过很短时间波包的不同部分频率会不同。如果将波包的不同部分看成一个个子波包,则它们各自有群速度,一般说这些群速度和频率有关(有色散),群速度色散就是描写这个波包群速度色散程度的量。从这里可知,当群速度色散不为零,则波包在运行中,其时间波形将会改变。只有当介质不很长,光脉冲通过介质时,波包的变形还不明显,或者波包的带宽很狭,色散导致的波包扩散不明显,这时包含群速度色散的项便可略去,非定态耦合波方程就成为

                  (18)

此外如果更接近实际些,则还要考虑傍轴的非平面准单色波,例如高斯光束,或在光波导与光纤中运行的波则方程应写成

                    (19)

可以按不同的情况选用合适的耦合波方程。当然处理愈接近实际,则方程的求解也愈复杂和困难。

\begin{exercise}
⒈  当介质中输入 两束光时,写出所有可能频率的三阶非线性极化强度的表示式.(已知 负频成分不写.\\
⒉  在介质中加一恒定电场  再输入一束频率为ω的激光,写出频率为ω的二阶非线性极化强度和三阶非线性极化强度的表示式。\\
⒊   证明在Kleinman 对称性成立时,独立的二阶非线性极化率元素由27个减少到10个,三阶非线性极化率元素由81个减少到15个。\\
4.   证明各向同性介质,三阶非线性极化率  有下列关系:
      1111=2222=3333
      2233=3322=3311=1133=1122=2211
      2323=3232=3131=1313=1212=2121
      2332=3223=3113=1331=1221=2112
      1111=1122+1212+1221
      其余元素为零. \\
5. 当介质中同方向输入 两束光时,求频率为 的二阶非线性极化波的波矢及频率为 的三阶非线性极化波的波矢。当两束光反方向输入时,又如何?\\
6.  设频率为 的平面单色波,其传播方向 与z 轴成 角, 证明沿z 方向的耦合波方程可写成    \\
7.  设频率为 的平面单色光波使得介质产生线性吸收,证明对于此光波的耦合波方程可表示为      
式中 是介质在 处的线性吸收系数, ,
而波矢 。 分别是线性极化率的实部和虚部。\\
     8..证明群速度和群速度色散可由折射率的色散公式 (Sellmeier 公式)来求得,即
                             


\end{exercise}
习题
