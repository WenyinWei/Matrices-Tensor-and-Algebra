\section{折射率随光强改变效应}
线性光学中吸收系数和折射率是介质的参数,它们与光强无关,但是当光强提高时,介质对光有非线性的响应,导致吸收系数和折射率将随光强而改变,由此产生许多新的非线性光学效应。本章只讨论透明介质中折射率随光强改变效应。设光与介质原子不发生共振吸收。首先唯象地定义非线性折射率,并讨论产生非线性折射率的几种重要的物理机制。然后介绍几种折射率随光强改变的效应,它们在光电子学和光子学中有重要的应用。
\footnote{本章节内容几乎全部从高健存老师于清华大学 2019 年秋季学期开设的非线性光学课程讲义中整理,引用请注高健存老师的名。}

\subsection{非线性光学折射率}
 一平面单色波通过各向同性介质,由于非线性光学效应, 将产生与输入波同频率的非线性极化,最低阶是三阶。

设输入光为沿x方向偏振的平面单色波

                                  (1)

在ω1 处的三阶非线性极化是

               (2)

作为非线性效应的波源,它对原输入波的影响由耦合波方程决定

                               (3)

式中有效非线性极化率是 。对透明的或弱吸收介质,χ(3)是实数,要解上式 可令 ,代入上式,并将等式两边实部与虚部分别相等,得到

                  (4)
                                                             
可见频率为 的三阶非线性极化只影响输入波的相位,不改变其能量。积分得 ,所以可写

                                      (5)

这里令 是实数, 是由于非线性引起的附加波矢(也就是附加相位),所以,场强的空间部分应为

                                                 (6)

若令                                             (7)

则折射率的非线性增量

                                       (8)

式中n 2  叫做非线性折射率

                                (9)

因此在非线性介质中, 折射率为

                                                          (10)

即产生了与光强有关的折射率增量。由于n 2和三阶非线性极化率有关,而在非共振时的三阶非线性极化率极小,因此这一折射率的非线性修正要在强光下才显著。可利用单光子或双光子共振来增强其效应。 此外,除考虑三阶贡献外,更高阶非线性极化也会对非线性折射率有贡献。
强光在介质中傳播不仅对自身相位有影响,可称为自作用, 而且对其它光波的相位也有影响,称为互作用。 为简单起见, 设二个平面单色波ω1,ω2共线输入介质,且有相同的线偏振.  它们的三阶非线性极化为

               (11)

式中自作用和互作用有效三阶非线性极化率分别是

                                     (12)

对 有类似的表示式。 将它们代入耦合波方程

                      (13)

类似以上单光束输入的解法,可得

                  (14)

由式可见,由于第二波的存在, 第一波的非线性折射率不仅由自身光强决定,也由第二波的光强决定。 引入第一波的自作用和互作用的非线性折射率,则第一波的折射率增量可表示为

                               (15) 

                     (16)

同理,对ω2 的光,有折射率的增量为

                             (17)

其中第二波的自作用和互作用的非线性折射率

                   (18)

当 I1>>I2 时, 第一波主要为自作用的贡献, 第二波主要为互作用的贡献,它们的折射率分别为: 

                                   (18)

即弱光的折射率受强光控制。

以上讨论带有唯象的性质,其实有多种产生非线性折射率的物理机制,它们会产生大小不同,响应快慢不同的非线性折射率。     

\subsection{透明介质非线性折射率的物理机制}
    
几种机制的特性可列表如下

机制	n2 (cm2/W)	χ(3) (m/V)2	响应时间(sec)
电子云畸变	10-16	10-22	10-15
分子取向	10-14	10-20	10-12
电致伸缩	10-14	10-20	10-9
布居数饱和	10-10	10-16	10-8
热效应	10-6	10-12	10-6-1
光折变效应	大	/	与光强有关

\subsection{光学克尔效应 (OKE)}
1875年克尔发现将恒定电场加在硝基苯上会使入射光的偏振态改变,这就是克尔效应.实际上它是一种三阶非线性光学效应。 1910年郎之万预言用光场可以感生克尔效应,称为光学克尔效应,但是光场要足够强才能观察到。激光出现以后,1964年Mayer和Gires用调Q红宝石激光器(55ns,140mJ,λ=694nm)在CS2,硝基苯CCl4等液体中首次在实验中实现光学克尔效应。

设一强泵浦光沿z 方向射入各向同性介质(常见的是分子液体),其偏振方向沿x 轴,另有一弱检测光共线输入,但是其线偏振方向与泵浦光偏振方向成任意角。由三阶非线性效应,强光将在介质中产生双折射,并对检测光的折射率产生影响。这时检测光的三阶非线性极化可写成 (由于是弱光,忽略自作用项)

                 (1)

代入耦合波方程

                 (2)

仿照第一节做法,得到检测光x,y 偏振分量的总折射率分别为

                                            (3)

式中由强光引起的检测光折射率的增量为

                    (4)

由于 ,所以x, y二个分量的折射率不同 (其不同是由光产生的所以叫光致双折射)。折射率的差值是                            

(5)

B(ω2)称为光学克尔系数,它是物质参数

                          (6)

经过L 长的介质,检测波二个分量的相位差为

                                       (7)

当δΦ 为0 ,2π的整数倍时,检测波的偏振态不变, 当δΦ 为π的奇数倍时, 检测波的偏振态旋转到与x轴对称的方位,特殊情形是检测光与泵浦光偏振方向成450的夹角时, 经过L长,检测光偏振态旋转900。 当 δΦ 为其它值时,输出为撱圆偏振,特殊情形是圆偏振。

典型的OKE实验装置如图 
           
利用OKE特性,可用强光对信号光进行相位调制,这在光电子学中很有用。尤其是有机分子液体,如CS2, 苯,氯化苯,甲苯等,它们是各向异性分子,分子取向是非线性折射率的主要来源。其弛豫时间在ps量级,所以各向异性分子液体具有快速反应的特点,因而得到广泛的研究。相应的器件如光学快门,全光型超快光开关,光学双稳态器件等也得到发展。当材料只有电子云畸变的机制时,可用光学克尔效应做极快速光开关。例如[ ]

除此以外,OKE 还用于测量物质参数包括分子感生的极化率,三阶非线性极化率以及在光作用下电子的和各种核运动的弛豫速率等。这些参数或者与分子的结构有关(例如可获得分子键结构细节的信息),或者与分子的周边环境有关,由此可得到分子可能的运动模式和弛豫通道的信息。这对于研究和改进快速光开关器件的性能也是必要的。

以上在讨论中,我们实际上假定是稳态或准稳态情形,即激光脉冲的时宽大于介质形成非线性折射率的时间或它的弛豫时间。然而产生光学克尔效应的机制不止是分子取向,也可能同是由电子云畸变,电致伸缩或其它机制。对不同的机理这个弛豫时间的大小是不同的。所以采用不同脉宽的激光可以将它们区分开来。特别是泵浦光采用超短脉冲(ps,亚ps 甚至几十fs脉冲),可以作高时间分辨的OKE研究,获得各种快速弛豫过程的信息。强的超短光脉冲使介质产生非线性响应,包括电子的以及核的响应,由于惯性,当激发光消失以后,介质的响应并不立即消失,而是按不同的机理以不同的弛豫速衰减。在这个衰减的过程中,以另一个也是超短脉冲的弱检测光去检测介质的响应,只要检测光相对于激发光有不同的时间延迟,就能得到介质各种机理引起的不同弛豫过程的信息,包括弛豫速率和弛豫方式等。这就是所谓时间分辨的泵浦-检测实验装置的基本思想。因此用ps-fs光脉冲的OKE实验装置不同于ns光脉冲的情形。目前广泛应用的光学克尔效应的外差探测技术(OHD-OKE)就属此类。它的装置如图
           
\subsection{椭圆偏振光的自作用}
光学克尔效应是利用强光在介质中产生人工双折射,通过互作用改变弱检测光的偏振。强光也可以通过自作用改变自身的偏振,但是必须是椭圆偏振光才能测到其偏振态随光强的变化。与光学克尔效应类似,它可用于测量物质参数。

为了方便的表示椭圆偏振光产生的非线性极化,我们先将各向同性介质的三阶非线性极化表示成与任意偏振态的场强的关系。已知各向同性介质的三阶非线性极化率81个元素只有21个不为零,其中独立的元素只有三个。因此可将三阶非线性极化率的任意元素用三个独立元素表示为
                          
(1)

将它代入三阶非线性极化

    (2)

就可以得到各种情况的矢量表示式。例如对自作用, ,此时有 ,即只有二个独立元素,所以由(1)(2)式可得

            (3)

或者写成矢量形式

                  (4)

此式叫Maker-Terhune表示。这里的系数 。(4)式中光场的偏振不限于线偏振,也适用于对其它偏振。由此可见(4)式右边的二项具有不同的贡献。于不同机理产生的自作用,它们的A,B系数的相对大小不同,这在折射率随光强改变的测量中有利于分辨出主要的机理。
对于椭圆偏振光,其电场的正频部分可表示成

                         (5)

引入圆偏振的单位矢量

                                   (6)

其性质是                        (7)

所以椭圆偏振光的电场又可表示成(正频部分)

                                   (8)

式中                             (9)

(8)式表示,一个椭圆偏振光可以分解成二个振幅不等的反向的圆偏振光。将它代入自作用的极化公式(4)便得到

                   (10)

如果将极化也写成椭圆偏振形式

                                         (11)

则其振幅                           (12)

下面我们证明左旋和右旋偏振两部分的极化分别产生不相等的非线性折射率即形成双折射。为此写出总极化(正频部分)

                                         (13)

其中线性极化是各向同性的。由电感强度

                   (14)

则有                                       (15)

其中                                         (16)

由(15) 可得到折射率为

                                              (17)

式中线性折射率 。 所以得到在强光下折射率的增量是

                                        (18)

它表明椭圆偏振光的左旋和右旋部分的非线性折射率是不相等的。它们的差值是

                                  (19)

由于椭圆偏振光左旋和右旋部分的振幅不相等所以这两部分将产生相位差,由(9)式可见椭圆偏振光振幅 的相对相位在传播中发生改变,导致偏振椭圆的旋转。在样品长度上测量对于输入光的旋转角,便可得到系数B,即三阶非线性极化率的某个分量。

    但是对于圆偏振光或线偏振光,自作用不会改变它们的偏振状态。

\subsection{光学相共轭波 (OPC)}

\subsubsection{定义}

已知单色光波为
                                            (1)

其中                   (2)        

对于非均匀介质,折射率和波矢与空间坐标有关,所以(1)式不是平面波。
定义它的相共轭波为

                                       (3)

      其中                                (4)

式中a 是任意复常数

如果原单色光波(1)是线性各向同性非均匀介质中波动方程的解

                                     (5)

则它的振幅满足

                             (6)

当介质是无损耗的,即介电常数 是实数时,很易证明,它的相共轭波也是此方程的解。因此只要边界条件合适,在线性非均匀的透明介质中,一个光波和它的相共轭波可以同时存在。相共轭波的特点: ⑴相共轭波的振幅处处与原波成正比(且差相同的相位),⑵传播方向与原波相反,⑶过同一点的波阵面与原波相同,⑷它是原波的时间反演.

产生相共轭波的装置叫相共轭镜, 相共轭镜与普通镜是很不相同的。

\subsubsection{用简并四波混频(DFWM)产生相共轭波}

相共轭波的产生有多种方式,这里介绍与折射率随光强改变有关的一种常用的方法即光路特别的简并四波混频。
                                     
图中1, 2为同频率ω 的泵浦光, 共线反向输入于各向同性介质(如克尔介质)中,即它们的波矢有

                                                        (7)

将第三个同频率的波(信号波) ( )与之有夹角输入。为简单处理起见,可令三波波矢共面且有相同的线偏振。由四波混频可以产生同频率的第四波,其相应的三阶非线性极化是

                    (8)

上述三项中第一项极化波波矢 ,它与信号波的反向波波矢相同: ,且振幅与信号波的振幅的复共轭 成正比,因此是相共轭波的波源,其余二项的波矢在其他方向,显然不能自动满足相位匹配,故此略去。

如果考虑各个波由强泵浦光引起的的自作用和互作用,则以上四个波的三阶非线性极化应写成
                
(9)

在假定泵浦光无衰减的近似下,上述泵浦光源项的自作用只是改变其相位,即

                             (10)                          

同样,


               (11)                         

因此     , 即与r 无关.

下面写出 的耦合波方程

                      (12)

或写成

                                              (13)

式中

              (14)                                 

令   ,得

                                                    (15)

边界条件为 已知,得解为

                           (16)

通常 ,则有

       (17)

由(17)式可见,信号波和DFWM所产生的相共轭波在传播中都得到增大(见图),自然它们的能量皆来自泵浦光。相共轭波在 处 的振幅和信号波振幅成比例,但比例系数与介质的选择,长度以及泵浦光强有关。可以定义信号光的透射率T 和反射率R

         (18)
        
由此可见反射率可以大于1,即相共轭波的振幅可以大于信号光,相共轭镜有放大信号光的作用。反射率R>1的条件是

                                                     (19)

当 则形成振荡,。

\begin{example}
     已知CS2在泵浦波长 处的 ,折射率是 ,液体池长 ,要有明显的相共轭波发生,则泵浦光强 应是多高?
由反射率公式(18)可知,R大小与K有关,这里K由(14)式决定。明显的相共轭波发生在 ,即(19)式的范围。可取 ,当L=0.4m则 ,由(14)式可以算得
    
调Q Nd:YAG激光单脉冲能量10mJ,15ns,聚焦后光斑直径小于5mm其峰值功率便可满足要求。
实验装置:
\end{example}

\subsubsection{DFWM与折射率随光强改变的关系}
    

DFWM产生相共轭光可看成光受相位光栅衍射的结果.与之有关的相位光栅有三个. 
\begin{enumerate}
    \item 泵浦光1与信号光相干,由非线性效应形成折射率光栅, 泵浦光2经此光栅衍射成相共轭光,
    \item 泵浦光2与信号光相干,形成折射率光栅, 泵浦光1经此光栅衍射成相共轭光
    \item 泵浦光1与泵浦光2形成空间均匀分布的并以2ω 频率振荡的折射率区域, 信号光受它散射成相共轭光。
\end{enumerate}

   
这一分析方法的好处是比较直观,例如很易判断,相共轭光质量好坏主要取决于折射率光栅的质量,因此对输入光的相干性和稳定性有要求。由于光导致折射率光栅的机理不同,形成光栅的驰豫时间也不同,用泵浦-检测实验便可区分它们 ,由此可研究产生相共轭光的机理。此外改变输入光的偏振可以确定哪个折射率光栅起作用,也可得到介质非线性折射率的数值,预示相共轭光的偏振等。  

\subsubsection{应用}

①自适应光学②精密光学测量③激光器④光电子器件⑤光信息处理等

\subsection{自聚焦和自陷阱}
考虑光束的有限截面,则光的自作用会导致波前的空间变化,产生自聚焦, 自散焦和自陷阱.。
自聚焦的发生是由于介质有正的非线性折射率系数( ),而有限光束的截面上光强具有非均匀分布。例如高斯光束,在光轴上有最大的光强,离开光轴光强指数衰减。由于折射率与光强有关,在光轴上折射率最大,在边缘折射率小,介质便形成正透镜将光束聚焦。聚焦作用加强了光强和折射率的非均匀分布,使正透镜作用更强,进一步聚焦光束,最终形成焦点。

横截面受限制的光束必然要产生衍射,截面愈小,衍射愈大。所以有限光束的自聚焦的过程也是与衍射相竞争的过程,衍射作用小于非线性折射率的汇聚作用则产生自聚焦,反之则不发生自聚焦。

理论上由光的传播方程处理。首先讨论有限截面,则光的自作用会导致波前的空间变化,产生自聚焦, 自散焦和自陷阱。设激光为单色的傍轴光,可以写成

              (1)

它在激光频率ω 处产生的三阶非线性极化为

                            (2)

代入旁轴光的耦合波方程

                            (3)

引入非线性折射率(见§4.1)

                                               (4)

则方程为

                                 (5)

如果输入为高斯光束,在旁轴条件下有解析解。对复杂形状的输入光,可对上面方程作数值计算来研究空间波形的变化。 

设输入高斯光束 (束腰位置在输入处z=0, 束腰半径为w0)的振幅是

                               (6)

计祘结果表明,存在一个临界光功率

                                                           (7)

式中 。此临界光功率与光束半径无关,和介质的非线性折射率有关。 当激光功率P < Pcr 则不产生自聚焦,光束因衍射而发散, 当激光功率P > Pcr, 产生自聚焦。当 P = Pcr.,光束既不发散也不会聚,即处于自陷阱状态.。此外要发生自聚焦,介质的非线性折射率n2 必须为正值。

计祘结果还表明, 当光功率大于临界值时,自聚焦的焦距为

                                                        (8)

激光功率超过临界值之后,自聚焦的焦距与激光功率有关,功率愈大焦距愈短。在计祘中假定高斯波形在传播中不变,故计祘结果是无象差的。如用数值计祘,放松旁轴的限制,就可得到有象差的更精确的结果。这时临界光功率和焦距分别为

                                  (9)

由(7)式可以估算临界光功率。对于克尔介质(如分子液体)可令n2~10-14 cm2/W, n0~1, λ0~0.5μm, 得Pcr~35Kw, 对 更小的介质如气体,水或玻璃, Pcr 会更高达到MW量级,因此需用大功率脉冲激光才可产生自聚焦。至于有吸收的介质,由热效应产生的非线性折射率很大,所以功率较低的cw激光也可产生热自聚焦。通常自聚焦的焦距zF约为几十cm。焦点处光斑大小理论上为零, 实际上在μm量级。

光在自聚焦之后的行为是个复杂的问题。.这是因为当光束聚焦到很小的区域时,极强的光功率密度导致自相位调制以及其它非线性过程的发生,如受激拉曼散射,多光子吸收和电离,光击穿等。这些过程要消耗光能量,所以光的进一步会聚,不会使折射率无限增大,而使自聚焦自行终止。但是当激光功率超过临界光功率很多时,聚焦后的光经常变成细丝状,单横模激光形成单一光丝,多横模激光形成多个光丝。细丝约为 粗细,延伸的长度与光功率和介质的性质有关,可以从几厘米到几十米。

临界光功率的存在可定性分析如下。当非线性折射率n2>0时,截面有限的激光束在介质中产生的折射率比四周的折射率要高,可以看成在介质中形成直径为d的〝光纤〞。 光纤内外的折射率分别为 ,由于全内反射则输入角大于全反射临界角θc的光将陷在光纤内

                                                     (10)
                                                     
当θc<<1 有                        (11)

另一方面, 直径为d的光束要衍射, 衍射角为 。当 ,则所有 的光将全部作全内反射,即光束自陷阱。由此条件,得 。将δn=n2I,代入式中,并利用光功率 ,便得临界功率 ,此式与前面得到的公式相近。

以上分析是稳态情形,实际上自聚焦的产生多用脉冲激光,这时依照激光脉宽τ 与非线性折射率的响应时间T 的相对大小,分为准定态和瞬态二种情况。

(1) 是准定态情况,这时可以运用稳态的结果,只是现在光强是时间的函数,所以自聚焦的焦点位置也随时间而变,即焦点是运动的。例如调Q激光脉宽在10ns量级,用于许多非对称分子的液体(克尔介质)或者调Q激光与锁模激光作用在以电子响应为主的介质。

其焦点的运动可在稳态焦距公式(8)中令光功率P=P(t-z/vg) 即为推迟时间的函数。这里 是脉冲的群速度。由(8)式可见 运动的焦点有两个,一个径直向前,另一个先反向移动到最短焦距处然后再转向前。如图所示。只是它们都在一束光丝内运动。这时如前所述,可能伴有其他非线性光学效应出现。
       
(2) 是瞬态情况,这时非线性折射率增量  不能瞬时地跟随光强 而变。当用锁模激光(脉宽在100ps~100fs)作用于分子取向类型的液体,或脉宽在几十fs 的锁模激光作用于原子气体,水或玻璃等以电子响应为主的介质就属于此类。与自作用相应的耦合波方程应采用旁轴光束的非定态耦合波方程,即(参见§1.7 公式(17))

                   (12)

上式中已忽略介质的群速度色散。右边的波源取自作用项,则有

                          (13)

式中 皆是 的函数。

对于分子取向类型的液体,其非线性折射  由德拜扩散方程决定, 

                                      (14)
                                      
其中源项是 。此式要与非定态的耦合波方程联合求解。一般要用数值方法计算。定性地说,脉冲在介质中传播时,它的前沿部分不产生自聚焦,因为对其前沿部分介质不能快速响应,但由此建立的非线性折射率却使脉冲的后沿受到聚焦。所以光脉冲直径最终发展成喇叭形,自聚焦区限制在脉冲的后沿。这一喇叭形脉冲由于衍射将在横向扩展,最终形成丝状,其尾部逐渐减弱。这些预示已在实验中观察到。由于超短脉冲的的峰值功率很高,易于产生自聚焦,所以容易在实验中出现。

通常自聚焦的产生是有害的,它限制了激光的输入功率,如激光晶体的热透镜效应,拉曼移频器拉曼激光器中的光损伤等。 但它也可被利用获得高强度的光和增加相互作用长度。

自聚焦现象是衍射和正的非线性折射率联合作用的结果,当这两个作用完全抵消,则光束既不收缩也不发散,长距离陷在由它建立的"光波导"中,这就叫自陷阱。不过它并不稳定,很容易受光强的起伏或其它干扰因素的作用而破坏。

\subsection{自相位调制 (SPM)}
前面我们研究了有限光束因非线性折射率而产生的效应。这里我们再研究超短脉冲(即时间受约束的光)因折射率随光强改变所产生的效应。即超短脉冲在非线性群色散介质中的传播。实际上空间的约束和时间的约束通过非线性折射率是相互关联相互影响的,不过为了简化讨论,我们将时空问题分开。这里假定输入光是平面波但其时间波形是脉冲形式。这时要用非定态耦合波方程(见§1.7(16)式) 

\subsubsection{线性有色散介质中光脉冲的传播}
在研究超短脉冲在非线性介质中的行为之前,作为基础首先要了解超短脉冲在线性有色散介质中的传播特性。这时,非定态耦合波方程(见§1.7(12)式)的右边令 ,得到

                (1)

式中  是群速度, 是群速度色散。

作变量置换
                                       (2)              
新坐标系叫推迟时间坐标系,在此系内,犹如在以波包的群速度跟随波包一起运动的系内观察波包的运动。方程(1)可化成 

       + =0                              (3)

如果假定群速度色散可以忽略, 就得到

        =0                                      (4)

它表示在推迟时间坐标系内脉冲波形不改变。还原到实验室坐标系则表示一个波形不变的脉冲以群速度vg 运动,即到达z 的时间有推迟, 推迟时间为z/ vg 。

如果群速度色散不为零,则方程(3)可以用傅里叶变换求解。为书写简单起见,以下省去变量的撇号,但仍表示在推迟时间坐标系内。 设方程(3)的解为 。将它作傅里叶变换

                                        (5)

代入方程(3)得到

                                       (6)

其解为   

                                     (7)

这里 是z=0 输入处光脉冲包洛 的傅里叶变换。(7)式表示如果输入脉冲的一个频率成分已知则经过一段距离这个频率成分便可知。将所有频率成分相加(作逆变换)即可求得脉冲在时域中的波形,即方程的解

                              (8)

为具体起见,假定输入脉冲的时间波形为高斯型

                                               (9)

 是波型参数。显然,脉冲的功率宽度(半高全宽)    。  (9)式的傅里叶变换为

                                    (10)

因此它的频带功率宽度(半高全宽)                                 (11)

高斯型脉冲的脉宽与带宽乘积

                                              (12)

将 代入逆变换公式(8),便得到任意z 处的波振幅

                                    (13)

光强为  

                                      (15)

式中参数                        (16)

由(15)式可见,高斯脉冲经过线性有色散的介质,仍然是高斯脉冲,但是脉冲的宽度有变化。传播到z 处的脉宽 (FWHM)为

                                      (17)

可见,不管群速度色散k〞是正或是负, 脉宽将随z 而增大 。这就叫波包扩散。由此可知群速度色散与脉冲变形有关,比它更高级的色散参数如 等也决定脉冲的变形。

可以定义特征长度

        ,                                    (18)        

当z<zo可以近似认为波包还未扩散,反之则应考虑群速度色散的影响。例如透明介质 ,对ps脉冲zo可为10m。对10fs 脉冲zo仅为1mm。

脉冲传播到z 处的谱强度由(8)得到

                                      (19)

它与z无关,因此线性介质中虽有色散,脉冲的谱线宽度在传播中不改变。

脉冲宽度增大(波包扩散)是因为介质的色散造成不同频率成分有不同的相速度,线宽不变是因为线性介质不会有频率成分的改变,而且各成分之间没有能量交换。

需要注意的是脉冲的谱线宽度虽然不改变,但由(7)式可见,其不同频率成分有不同的附加相移 ,它显然来源于介质的群色散。正是不同的附加相移才导致不同的频率分量有不同的相速度,结果使波包扩散.。所有频率分量有相同相位的脉冲称为傅里叶变换极限型脉冲。例如由(9)表示的输入高斯脉冲便是傅里叶变换极限型脉冲。但是当它进入有群色散的介质后,它就不再是傅里叶变换极限型脉冲,因为不同频率分量有了不同的附加相位。此种脉冲就称为频率啁啾脉冲。为了更加了解啁啾脉冲的性质,我们回到时域中考察。这时可将脉冲波形写成
                                          
(20)

对高斯型脉冲,振幅的模是

                                    (21)

有介质色散引起的附加相位是

                                  (22)
                                  
式中                                           (23)

固定空间一点看,它的相位随时间改变 (这里,与t 2 成比例)

如果将整体准单色光写出
        
令它的的正频部分与时间有关的相位因子是 ,则 便是

                                          (24)

引入瞬时频率                                             (25)

则有

                                                  (26)

它表明定点观察此脉冲,则其瞬时频率将随时间而改变。  当β>0, 瞬时频率将随时间而增大,反之则随时间而减小。 β>0,即k″ >0, 群速度色散为正, 反之则为负。分别称这样的脉冲为正(负)啁啾脉冲。 如果瞬时频率随时间线性变化就叫线性啁啾。啁啾脉冲是非傅里叶变换极限型脉冲。非傅里叶变换极限型脉冲的脉宽与线宽的乘积总大于傅里叶变换极限型脉冲。

如前所述,一个傅里叶变换极限型脉冲在线性有色散的介质中会变成啁啾脉冲, 频率啁啾的正负由k″决定。 反之,对一个啁啾脉冲也可用适当的线性色散介质对其作相位补偿,使之成为傅里叶变换极限型脉冲,从而使脉冲宽度得到压缩。

\subsubsection{非线性无色散介质中光脉冲的传播}
超短脉冲在非线性有群色散介质中的传播由非定态耦合波方程(见§1.7(16)式) 决定。为了强调光强对自身相位的影响,我们先设介质无群速度色散。此时方程为

           (1)

右边的源项中的非线性极化是自作用项,即

                     (2)

这样的表示对应于瞬态响应。将其代入(1)式,并作坐标变换,换到推迟时间坐标系,即令
                                              
(3)

于是得到

                               (4)

注意为书写简单起见,以下式中所有  的撇号都已略去,但须记住我们是在推迟时间坐标系中讨论。式中  是自作用非线性极化率。设它为实数时,可令

                                                     (5)

 代入方程(4),并令方程两边实部和虚部分别相等,则有

                                   (6)

方程积分得

                    (7)                       

                      (8)

上式中 是起始脉冲振幅,可取为实数。式中t 为推迟时间。(7)式表明脉冲以群速度 传播,在传播中波形不变,能量也不变,但获得了附加相位 。由 可见,这一附加相位产生了与时间有关的非线性折射率。这里忽略了介质对光的响应有一定的时间,所以只适用于准定态情形。由于出现了与时间有关的附加相位,因此光波的频谱将发生变化。可以对光场作傅氏变换,利用(5)式就有

         (9)      

式中起始光场 可以取各种脉冲形式,但即使是高斯脉冲,上述积分也比较复杂。以下给出定性的结果。 对一般钟形脉冲,其功率谱特征如下: 首先, 如果脉冲时间波形是对称的,则脉冲的附加相位也是对称的,故功率谱对ω0 也对称,相反脉冲时间波形不对称则功率谱对ω0 也不对称。(对于非线性折射率滞后于光脉冲的情形,不对称是显然的)其次,频谱比输入时有很大扩展。如果输入脉冲的线宽由其脉宽决定则经过自相位调制它的线宽将会大大超过起始线宽。 ns脉冲可展宽几十cm-1,ps脉冲可展宽几千cm-1。第三, 功率谱强度有半周期性结构,即频谱中有干涉条纹的结构。

      从时域来看, 由(7)式得到

                                                    (10)

它表明,传播中,不但光功率不变而且脉宽也不变亦即脉冲的时间波形不变。这是假定介质无群速度色散的结果。但由于其相位受光自身的调制,脉冲在传播中它频率将随时间而变,即已成为啁啾脉冲。可以考察如下: 写出电场强度的完整表示式

                   (11)

固定z0 来观察脉冲,其相位随时间的变化是

                                              (12)

瞬时频率为                          (13)

可见为啁啾脉冲。 以高斯脉冲为例,设

                                                  (14)

这里 是以高斯脉冲强度的 定义的半宽度。 代入(13)式得到瞬时频率为

                                          (15)

由下图(a)可见
 
对γ>0 (n2>0)的介质, 瞬时频率由 决定。它有二个极值 ,此极值由非线性折射率对时间的最大变化率决定。在这二个极值之间, 瞬时频率随t 近似线性增长,所以脉冲基本属于正啁啾脉冲。频带的宽度近似由二个极值之间的频差决定。自相位调制产生的带宽随输入光强和z 而增大,乃至会大大超过输入时的带宽。所以利用自相位调制是实现宽带光源的重要方法之一。 由上图(a)的瞬时频率分布还可看到,一个脉冲中,都有二个相同的频率成分,但它们有不同的相位关系, 因此在这些频率成分中,有的干渉增强, 有的干渉相消,这就是功率谱有半周期结构的原因。

当脉宽小于或与感生非线性折射率的弛豫时间可比拟时即为瞬态情形。这时 不仅取决于同一时刻的光场,而且依赖于所有比t要早时刻的光场,所以 的变化将滞后于光强。自相位调制 也如此,所以功率谱将发生新变化。虽有功率谱加宽以及干涉峰结构,但功率谱对输入频率 不再对称,谱的反斯托克斯成分延伸得少,能量也小,谱的斯托克斯成分延伸得多,能量也大,如上图(b)。(对 的介质,功率谱主要是反斯托克斯成分)

强光脉冲的光谱增宽在许多介质中(包括液体,气体和固体)都已经观测到,不仅在 大的克尔液体中也在 小的介质中测到。在一定的条件下上述谱的特征都已发现。但是随着研究的深入,特别是超快激光器的进展,由它产生的光谱的增宽远远超过上述自相位调制所预示的范围。即产生谱宽从紫外到红外,带宽为几百nm的超连续光谱。实际上如此反常增宽的谱的产生决不是一种机制可以解释。首先平面波的近似过于简单。很早就已发现空间的自聚焦对光谱的异常增宽有重要影响。实际上空间的相位调制和时间的相位调制是相互关联的,用自聚焦的成丝和运动焦点激励可以解释谱增宽的许多性质。除了自相位调制外,由于聚焦使强光在介质中产生等离子体,与光强有关的电子浓度会改变介质折射率,因而造成相位调制也是光谱异常增宽的重要原因。

其它如四波混频,受激光散射等都可能起作用。   

超连续光谱和白光激光器的研究是非线性光学的前沿课题之一,对光学的基础研究与应用研究都有重要意义。

以上讨论假定超短光脉冲在非线性无群色散介质中的传播。当介质的群速度色散很弱,或者介质的长度较短使色散效应来不及展示时,这一近似是合适的。然而当群速度色散很强,例如工作在反常色散区,或者介质与光的相互作用长度很长,例如在光纤中,则群速度色散不可忽略,这时必须研究超短光脉冲在非线性有群色散介质中的传播。我们将在第七章讨论它。

\begin{exercise}
    ⒈ 已知CS2  对波长1μm的光其线性折射率n0为1.60, 其自作用三阶非线性极化率  为 求它的非线性折射率n2, 若光强为1MW/cm2, 求折射率的增量Δn.。如此小的增量,用什么方法可以测出?\\
    2. (1)设分子液体中,非线性折射率 的变化由德拜扩散方程表示为
   
  式中 是分子取向的弛豫时间。若激光脉冲的光强 已知,求解此方程。
  (2)如果非线性折射率 的变化由以下方程决定
            
  式中 是与分子自由摆动的频率有关的参量。若激光脉冲的光强 已知,求方程的解。(可用格林函数方法)\\
    3. 在各向同性介质中,由基波产生三次谐波,其相位失配量是Δk=k3-3k1, 但考虑到折射率随光强改变,新的ΔkNL是多少?,当Δk=0时,相干长度如何表示(用小信号近似来讨论)\\
    4. 在OKE实验中,令检测光偏振与泵浦光成θ 角,证明探测器接收的检测光强是
             
  θ 角取何值信号最大?设检测光通过正交起偏和检偏器\\
    5. 证明在各向同性介质中,DFWM产生相共轭波时,与它相应的三阶非线性极化强度为
            
  式中  \\
     6. DFWM产生相共轭波实验可测反射率R, 证明当R<<1时, 测R可得三阶非线性极化率     \\
     7.已知苯  对波长1μm的光其线性折射率n0为1.49,其非线性折射率 ,求高斯光束腰处输入时的自聚焦临界功率,当束腰半径为500μm并且输入功率为临界功率1.5倍时的焦距,分无象差和有象差两种情形计算。\\
     8.  在无群速度色散的介质中,自相位调制将导致异常频谱加宽。设输入为高斯脉冲由§4.8(14)式表示,在瞬态响应下证明脉冲展宽为   ,在功率谱中红移部分的干涉条纹数近似是                                   
  
\end{exercise}