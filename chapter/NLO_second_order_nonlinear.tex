\section{参量混频过程(二阶非线性)}
一个或几个频率的光与介质的非线性相互作用可以产生新频率的光。输入光和产生光之间有能量交换,即它们之间是相互作用的。本章讨论二阶非线性光学效应中的参量混频过程。二阶非线性光学效应通常为三波相互作用,称为三波混频。此类过程都发生在介质的透明无损耗的频带内,此类介质多为无机晶体,有机晶体以及高分子材料,它们在宏观上必须不具有中心反射对称性。为了有效的产生新的频率成分,所有参与的光波波矢作为参量必须满足匹配条件,所以它们叫做参量混频过程。

本章将重点讨论:三波混频中的能量转换规律以及提高能量转换效率的途径,介绍几个参量混频过程以及发展中的新技术。
\footnote{本章节内容几乎全部从高健存老师于清华大学 2019 年秋季学期开设的非线性光学课程讲义中整理,引用请注高健存老师的名。}

\subsection{三波混频  曼利-罗(Manley-Rowe)关系}
在没有直流电场参与下,二阶非线性光学效应通常为三波相互作用,或称为三波混频。设介质中有三个平面单色光波,沿z方向(共线) 传播, 频率各为 通常其中二波由外界输入,第三波是新产生的。则它们可以表示和频发生,或差频发生,当 时也表示二次谐波发生。将它们的电场写成

    (1)

其中 是复数振幅,  是波矢 , 是偏振方向的单位矢量。在这些光波作用下,介质将产生以上频率成分的二阶非线性极化波,它们也沿z方向传播。

                       (2)

式中 表示极化波的包络, 是极化波的波矢。因为极化波不同于电磁波,所以它没有类似()的色散关系。这表示极化波传播中的相位不同于电磁波的相位。极化波的波矢要根据激发它的光波波矢来决定。设二个沿z方向传播的光波,频率为 ,在介质中产生一个频率为 的二阶非线性极化波,则此极化波可写成

                              (3)

由定义

             (4)

将(4)式两边的量分出相位因子则有

                (5)

比较等式(5)两边,得到

                                                  (6)        

即极化波的波矢为激发它的相关光波波矢之和。(注意当(4)式中有负频时,则它的波矢也应为负)。当光波波矢共线时,它们是代数和, 当光波波矢不共线,它们是矢量和,即  。一般说来, 极化波的波矢与同频率的光波波矢是不相等的。
利用(5)可将频率为 的光波的耦合波方程写成

                (7)                         

或                    (8)

式中定义

                   (9)                   

叫做频率在 处的有效二阶非线性极化。.注意在透明区它是个实数标量。

                                       (10)                                             
 叫做相位失配量,其重要意义将在后面讨论。同理,考虑三波的相互作用,可写出 光波的耦合波方程

                                          (11)

                                   (12)    

对无损耗介质,由非线性极化率的全交换对称性可证明

                   (13)    

在利用上述耦合波方程讨论各种具体的混频过程之前, 首先建立三波相互作用中能量转换的普遍规律, 即曼利-罗关系。为此,我们将上述耦合波方程化成光强的方程。已知一个频率为 沿z 方向传播的平面单色光波,若振幅为 则它的光强(即功率密度)为

                (14) 

   是介质的线性折射率。 将光强对z微分

                             (15)       

将耦合波方程及其复共轭方程代入上式,便可将光场的耦合波方程化为

                                  (16)                  

式中Im表示取虚部。将三式相加,得到

                  (17)                

上式表示在三波与介质相互作用中,尽管每一个光波在传播中它的光强在变,但是三波光强之和则是不变的(等于输入的总光强)。这反映参量过程的一个特点即光波与无损耗介质相互作用时,能量只在各光波之间交流,而不在光波与介质之间交换。介质不参与能量交换,仅起作类似"催化剂"的作用。

由上面三个光场振幅的耦合波方程还可得到一个关系即

                                   (18)

这里普朗克常数的引入是为了说明在相互作用中三个光波之间能量转换的一般规律,即沿传播方向单位长度上高频光的光子数减少(或增加)量恒等于二个低频光的各自光子数的增加(或减少)量。(18)的关系在三波混频中是普遍的,称为曼利-罗关系。


\subsection{二次谐波发生 (SHG) }
在无中心反射对称性的晶体中,输入频率为 的激光,由于二阶非线性效应, 晶体中会产生频率为 的光波,这个现象叫二次谐波发生。Franken等人于1961年用红宝石激光在石英晶体中首次观察到这个效应。以下用平面单色波作理论分析。

设基频波的空间部分为 (只写正频部分)               (1)

二次谐波的空间部分可设为                                          (2)
式中 分别是基频波和二次谐波的偏振单位矢量。这两个光波与介质的相互作用,产生在 处的二阶非线性极化是

             (3)

将它们作为光波的波源分别代入各自的耦合波方程,得到

                                    (4)

                                    (5)

式中n 是线性折射率,  是在 处的有效非线性极化率,同样有 ,注意作为标量,有效非线性极化率的大小不仅和二阶非线性极化率有关而且还和各个光波的偏振有关。由全交换对称性可以证明

                                           (6)

方程中        ,                                            (7)

叫二次谐波发生的相位失匹配量。利用边界条件即 是已知量,  ,求解上述方程。

首先求小信号近似解。假设基频光很强,产生的倍频光很弱,可以认为基频光近似无衰减,即令 。这时只须解倍频光的方程(5),并在方程的右边,将 代以初始值 。利用边界条件得到的解是

                                            (8)

二次谐波的光强为

                                 (9)

由此可见二次谐波较弱时它的光强与基频波光强的平方成正比,在传播中随z 的变化是:   。即倍频波与基频波沿传播方向其能量有周期性交换.  愈大, 交换周期愈短,二次谐波光强的峰值愈小(如图)。产生能量周期性交换的原因是二次谐波的波矢与同频率的极化波波矢不相等,因此在传播中相对相位周期性变化,导致二次谐波能量周期性变大与变小。
            
     定义相干长度为:
                                              (10)

当 时,能量由基波流向谐波首次达到最大。 愈小, 相干长度愈长,  当 为零,则相干长度无限大,此时 , 即能量单向地从基频波流向倍频波。  Maker等曾用红宝石激光经过石英晶片产生二次谐波研究其输出与晶片厚度的关系,  测定了相干长度。通常这个相干长度是 的量级。为了能有效的单向转换,使 是一个重要的条件,这个条件就叫做相位匹配。  

                                                      
定义二次谐波的转换效率为输出的倍频光强与输入的基频光强之比

                         (11)

则在小信号近似下,二次谐波的转换效率为

                  (12)

由上式可见,当晶体选定以后,要提高二次谐波的转换效率,首先是提高输入光强,因为二次谐波转换效率和基频光强成正比。其次要使得有效二阶非线性系数尽可能大,第三,尤其重要的是要满足相位匹配条件。可是当.转换效率提高时, 小信号近似就不成立了。这时要考虑基频波有衰减,所以要用两个耦合波方程联立求解。为简单起见我们只求满足相位匹配条件时耦合波方程(4)(5)的解。改写方程为

                                   (13)

式中令系数  。 求解时,设

                                   (14)

相应的光强                                                 (15)

                                                     (16)

代入耦合波方程(13),并将实部和虚部分离,得到

                   (17)

令                                        (18)

可得到                

或                                      (19)

因此有

          ( 是与z 无关的常数)                  (20)

利用通常的边界条件即起始处二次谐波为零, ,可知   。所以在 处, 。我们可以取 ,則方程(17)变成 

                     (21)                      

由此可得到        

或者                          (22)

在(21)式中,利用(22)将 表示成 的函数,最后解出

                      (23)                           

                              (24)                       

最后得到基频波光强为                    (25)                

倍频波光强为

                   (26)                    

式中定义二次谐波发生的特征长度为                    

                                  (27)                    

这时, 基频波和倍频波沿传播方向光强的变化如图所示。二次谐波发生的转换效率为

                                    (28)                        
                                     
当晶体长等于特征长度时,二次谐波发生的转换效率为58\%,注意SHG的特征长度和输入光强是有关的。所以转换效率也和输入光强有关。当晶体长度超过特征长度则绝大部分的基频波能量已经转给倍频波,增加晶体长度不会再明显高转换效率。L通常取在特征长度附近。当 则(28)便过渡到小信号近似式(12)( 情形)因此提高二次谐波发生的转换效率主要仍归结为三个因素:即相位匹配,介质的非线性极化率和基频光的输入光强。

\subsubsection{相位匹配的实现}
二次谐波发生的相位匹配条件是, 即                     (1)

即要求介质在倍频光和基频光处的折射率应当相等。满足这个条件的最常见的匹配方法是利用某些晶体的光学双折射效应。 各向异性的晶体具有双折射特性,即沿同一个传播方向的光波具有两个本征模,它们的电位移矢量互相垂直并与波矢方向垂直,两个本征模一般有不同的折射率。由于色散的存在,它们的主折射率是频率的函数,因此可以利用不同频率不同本征模的折射率在一定条件下可能使倍频光和基频光的折射率相等即达到相位匹配。依照基频光偏振方式的不同匹配方式可以分为Ⅰ类匹配和Ⅱ类匹配。Ⅰ类匹配是指输入的基频波具有相同的偏振,(实际为同一束光),Ⅱ类匹配是指输入的基频波具有正交的偏振(频率相同偏振不同的二束光)。以下以单轴晶体为例说明如何实现相位匹配。

  ㈠ 角度匹配

对单轴晶体要区分正单轴和负单轴两类.它们的折射率面如图(也见§1.5)

       
 相位匹配条件分别是

负单轴晶体的Ⅰ类匹配只能是 方式, 即输入的基频光为o光产生的倍频光为e光。 匹配角由下式决定:由条件 可得

                                        (2)

式中当e光的传播方向与光轴成 角时它的折射率是

                     (2 a)
                     
两式中 分别是o光和e光在不同频率处的主折射率,它们的值由该晶体的Sellmeier公式决定。如果满足(2)式的匹配角存在(即 必须是实数)则可实现该晶体的Ⅰ类角度匹配。
负单轴晶体的Ⅱ类匹配只能是 方式。即输入的基频光分别为o光和e光,产生的倍频光为e光。  匹配角由下式决定:由条件  可得

                                              (3)

其中方向角为 时e 光的折射率公式与(2a)式类似。只要满足(3)式的实数匹配角存在就可实现该晶体的Ⅱ类角度匹配。

正单轴晶体的Ⅰ类匹配只能是 方式,   类似讨论可知其匹配角由下式决定:

                     (4)                                      

正单轴晶体的Ⅱ类匹配只能是  方式, 其匹配角由下式决定

               (5)                                  

当折射率参数既满足Ⅰ类条件也满足Ⅱ类条件,则两类匹配都存在,当某个匹配条件不满足则这个匹配不存在。

    ㈡  温度匹配

角度匹配的缺点是对匹配角的要求很灵敏,同时有限光束在双折射晶体中有离散效应(walk-off)。一般情形e光的能量传播方向与o光有不同,即e光有离散角而且离散角的大小随光与光轴间夹角而改变,这导致这二束光在传播中产生空间的分离,从而使得转换效率下降。 特别是光束强聚焦时更显著。为使离散角为零,可令光束平行于或者垂直于光轴传播,这时离散角为零。由于此时角度已经固定,必须用其它参数来调节折射率,使其达到相位匹配。最常利用的是一些晶体的主折射率随温度而改变,因此可用调节晶体温度以达到相位匹配。 例如将基频光垂直于光轴输入负单轴晶体, 用控温炉调节晶体温度,在合适温度下就有可能达到相位匹配。

对负单轴晶体在Ⅰ类匹配时温度匹配的条件是

                                                     (6)

Ⅱ类匹配时温度匹配的条件是

                                 (7)

式中不同频率的主折射率与温度参数的关系由晶体的含温度参数的Sellmeier公式决定。
温度匹配

以上讨论的是单轴晶体中的相位匹配, 对双轴晶体的处理更加复杂。

实际上二次谐波发生的匹配条件可以有一定的放宽,即允许 有一个范围。允许Δk有一个不匹配范围,即满足 。也就是由相位匹配峰的半高全宽所决定的范围: (L是晶体长)

2Δk =0.886 π ×2/L=1.77 π/L                                      (8)

求出这一限度对于应用是很重要的。因为k与折射率有关,而折射率常是某些参数的函数,例如角度,温度,波长等。所以 有一个范围,也就是允许激光线宽,匹配角和匹配温度可以有一个临界范围。设折射率为 ,这里x 表示角度 ,温度T,波长 等。则可将 在相位匹配条件附近按参数x展开

                         (9)

这里xo表示参数取值在相位匹配时,故右边第一项为零。其余各项的系数可由Sellmeier公式求得。再由条件§2.2(11),就得到 的允许值为

                                          (10)

此式是假定(9)式右边第二项系数不为零,如果它为零则需取第三项,这时便有

                                     (11)

由此便求得匹配角,匹配温度和激光线宽的临界值。其中对激光线宽的限制就是考虑到实际激光都有一定的线宽,对脉冲激光,其脉宽和线宽的关系有 ,所以对激光线宽的限制就是对激光脉冲宽度的限制。如果线宽超过这一限制则其中一部分频率成分会因为不满足相位匹配而不起作用,使转换效率下降。

如果相位匹配的调节参数有较大的临界值就叫做对此参数的非临界匹配。例如单轴晶体温度匹配时传播方向与光轴垂直,此时离散角为零,而且光波输入角度的临界值与角度匹配时相比有较大的值,所以温度匹配是非临界匹配,角度匹配则为临界匹配。以上结果适用于平面波,实际情形激光应视为高斯光束,则以上讨论将要做适当修改。

\subsubsection{准相位匹配}
近年来,随着人工微结构器件研制技术的发展,产生了准相位匹配 (QPM)方法,并已成功用于二次谐波发生和其它三波混频。它对非线性光学变频器件的小型化有重要意义。

有一类二阶非线性晶体也是铁电晶体,如LiNbO3, KTP, LiTaO3 等。晶体在一定温度下生长时,它们会自发极化,即带有极性。自发极化使得晶体的光轴有了确定方向。极性反转则光轴也反转。这对线性极化率没有影响,即线性折射率不会因光轴的反向而改变,但会使二阶非线性极化率有些分量要变号,并导致垂直于光轴传播的光波,其有效二阶非线性极化率要变号,或者说, 二阶非线性极化经过极性反转的界面后它的相位会突然改变π ,即获得  的附加相位。如果采用适当工艺使之根据需要做成极性周期性反转的空间结构(如图),它就形成一维光栅,光栅的波矢由反转周期决定。当基频波和倍频波的波矢不匹配,即产生不为零的 时,利用光栅波矢来抵消 从而达到相位匹配,这就叫准相位匹配。
     
由二次谐波的耦合波方程

                                     (1)

式中 , 对于电畴的周期性反转结构这里的 已是z 的周期性函数。可设为方波函数,并将它展成付里叶级数,得到

                      (2)

是方波周期,(见图)。 是无周期时介质的有效二阶非线性极化率。 将(2)代入耦合波方程(1),就有

                           (3)

对 也有类似的方程。当方波周期合适,使上式级数中m=1 的项满足Δk-kL=0, 则上式右边除m=1 的项外,其它的项皆可忽略,因为只有m=1 的项补偿了相位不匹配量.而其余的项指数因子不为零,即相位不匹配。这时耦合波方程(3)右边只保留一项故有

                                             (4)

式中 (式中近似取n1≈n2),可见它与一般晶体相位匹配时的方程相同。这就是准相位匹配的由来。 准相位匹配的条件并不要求做到 而只要求Δk-kL=0,也就是在制造铁电晶体的极性反转时人为的使周期满足
                  即 ,                                      (5)

这里  Lc是相干长度.,通常相干长度是微米量级。所以只要层厚为相干长度并使极性周期性反转便可实现二次谐波的高效率转换。 同理,利用m=3 的项也可实现准相位匹配(叫做三级准相位匹配),它的工艺上的困难降低但效率不如一级准相位匹配高。

准相位匹配的加工工艺如图所示。

准相位匹配的优点是材料不受有无双折射的限制,因此适用于更宽的波长范围,可利用最大的非线性极化率的系数以提高转换效率,可利用半导体集成技术加工成波导形式,有利于和光纤及其它光波导相耦合等。所以它在光通信和光电子学的其他领域有重要的应用前景。其缺点是只适用于低功率的小型器件,对输入激光的线宽要求高等。

\subsection{有效非线性极化率}
对二次谐波发生,其转换效率与二阶有效非线性极化率的平方成正比。所以要选择有效非线性极化率高的晶体。于二次谐波发生有关的有效非线性极化率定义是

  (Ⅰ类) 
或  (Ⅱ类)           (1)

应当注意 不仅与二阶非线性极化率张量元素的大小有关而且也与基频光和倍频光的偏振态有关。这里 表示基频波不同的偏振态。要计算有效非线性极化率,我们首先将二阶非线性极化率张量加以简化。利用非线性极化率张量的固有交换对称性则张量元素有以下关系

                    (2)

因此可将下标压缩如下: 令 ,即将 与 对应。 原来jk共有9个值这时收缩为6个。它们的对应关系是
     
利用压缩的下标则有效非线性极化率可以表示为
           
 是由基频波不同偏振态分量组成的量,

     (3)                   

 是第一个基频波偏振矢量在j轴上的投影。将 写成列矢量就是

                                                  (4)                                                      
当基频波二个偏振矢量相同时,上式就对应Ⅰ类匹配。

在应用上常使用另一个二阶非线性极化率张量 ,它与 的关系是
 ,                                        (5)

可见 张量有18个分量。可表示为
        
由于空间对称性, 各元素之间的关系也反映在 上.例如KDP晶体属于 晶类,其非线性极化率 为
             
因此新的有效非线性极化就是

                           (6)                 

为了由(6)式计算有效非线性极化,可在晶体上建立主坐标系,标出光的传播方向单位矢量 ,它与坐标轴的关系是

                                      (7)
                            
对于单轴晶体,x3 就是光轴,因此o 光和 e光的单位偏振矢量分别是

                 (8)

例如对于负单轴晶体KDP, 当光在主坐标系内沿方位角 传播,利用它的非线性极化率 和(8)式代入(6)式便可算得晶体的有效非线性极化率是

Ⅰ类匹配:     

Ⅱ类匹配:     

可见有效非线性极化率不僅和 而且和 有关。在角度匹配时  取相位匹配角,在温度匹配时 取 。即使如此, 还有选择的余地。上例中Ⅰ类匹配时取 最佳,而Ⅱ类匹配则不存在。
   实际应用中对晶体材料的考虑不仅是它的非线性极化率,还要考虑其光学透明域,可以实现相位匹配的范围,晶体的光学损伤阈值,是否易于镀膜,是否容易潮解等等。

\subsubsection{峰值功率}

提高二次谐波发生转换效率的另一个重要因素是提高基频波的功率密度或峰值功率。当激光源的输出功率不变时可采用透镜聚焦的方法作空间压缩,脉冲激光可采用调Q或锁模技术作时间压缩,提高功率密度。
用透镜聚焦提高功率密度一方面受晶体光学损伤阈的限制,另一方面强聚焦使光束截面缩得更小,光束离散效应加大,相互作用长度减小反而导致转换效率下降。Boyd, Kleinman研究了激光在聚焦区产生二次谐波的转换效率,它可表示为
 
式中 是激光功率,L是晶体长度,w0是束腰半径,聚焦参量 是共焦参数),双折射参量 ( 是离散角),特别是引入效率减小因子 它与聚焦参量和双折射参量的关系如图所示。由此可见并非聚焦愈强转换效率愈高。

非线性光学常采用脉冲激光作光源,因为它可以获得高的峰值功率。近年以来发展的超短脉冲激光器,已成为研究非线性光学效应的有力工具,我们将在§ 介绍超短脉冲激光的二次谐波发生。

\begin{example}
    使用Nd:YAG脉冲激光器在KDP晶体上作二次谐波发生。如果满足相位匹配条件,设晶体长为0.5cm,当要求转换效率为50\%时求激光器所需的峰值功率。
\end{example}


与脉冲激光器的峰值功率相比,连续波激光器的输出功率要低得多。因此用连续波激光作倍频通常要采用谐振腔。有腔内倍频和外腔倍频两种方式。如图所示
           
腔内倍频是将倍频晶体放在激光腔内,由于腔内激光的光强约为腔外的100倍,因此大大提高二次谐波发生的转换效率。此种结构也相对的简单。问题是尽管激光晶体是均匀加宽介质,但激光腔是驻波腔,由于空间烧孔,所以是多纵模振荡。模式之间的竞争以及通过它们产生的二次谐波之间的干涉造成二倍频光的输出不稳定。解决的方法是消除或抑制空间烧孔,例如采用环形腔(单纵模运转),或吸收系数高但长度短的激光晶体等,也可以使很多纵模同时振荡,使微观的模式竞争"平均化"。

外腔倍频是将倍频晶体放在激光器外的另一个谐振腔内,当激光模式与这个谐振腔的模式相重时(腔共振),谐振腔内的光强大大超过输入的光强,便可提高二次谐波发生的转换效率。外腔倍频的优点是输出稳定,激光器不受倍频晶体的影响,但它的结构复杂,匹配困难。激光器要单模(环形腔),并且要稳频。为了使激光与倍频腔模匹配,倍频腔长需用伺服系统与基频波的波长相关联等。

\subsubsection{超短光脉冲的二次谐波发生}

近年来,超短光脉冲激光器有很大的发展。由于它可以产生很高的光峰值功率密度,因而易于产生非线性光学效应,加之超短光脉冲激光有很宽的可调谐范围,因此在参量混频中有广泛的应用。通过频率变换构成可调谐的激光光源,是物质结构和光谱研究的有力工具。

在以前我们已经指出,当激光脉冲的宽度 接近或短于介质的响应时间,光和原子的参数是时间的函数,相互作用将随时间有迅速的变化。发生在透明介质中的参量混频主要由电子云的畸变产生,所以介质对光的响应时间极短,约为 。脉宽在皮秒至飞秒之间光脉冲称为超短光脉冲,通常由锁模激光器产生。

超短光脉冲与通常的纳秒激光(如调Q激光)在参量混频中的不同在于后者可用定态方程描述,而前者必须用非定态方程描述。因此用超短光脉冲作参量混频时,在提高转换效率的诸因素中还须加上与时间有关的因素。它们是(1)不同群速度之间的匹配(2)群速度色散引起波形变化(3)自相位调制对相位匹配的影响。其中关于(3)将在第  章中介绍。

以下讨论超短光脉冲的二次谐波发生。

建立基频波和二次谐波的非定态耦合波方程

            (1)          

式中 分别是群速度和群速度色散, 是与各波的有效非线性极化率有关的耦合参数。由方程可见,基频波与倍频波以各自的群速度传播,经过一段距离,这两个脉冲必将分开,分开之时他们的能量再也不会有效地转换。所以可以找到最佳的晶体长度使能量转换最为有效。在这个长度以内,基频波与倍频波始终重叠。即这一长度一定要满足

                                                         (2)

这里 是激光脉冲的宽度,它与频带宽度有关系 。(2)式叫群速度匹配关系。如果激光脉冲宽度(或是带宽)已定,则晶体长L不能大于(2)式规定的值,因为超过此值则基频波和倍频波将因二波不能在空间重叠(即时间的walk-off效应)使效率不能提高。实际上(2)等价于以前所说的对激光带宽的临界值的要求(见习题)。带宽中超过临界值的部分对二次谐波的发生是无效的。

群速度色散对转换效率的影响:有群速度色散时光脉冲在传播中要扩散并产生频率啁啾(即脉冲的不同时间段有不同的载波频率),前者使得峰值功率下降因而降低转换效率,后者导致不完全的相位匹配,所以用超短光脉冲做混频时,应当尽可能减少群速度色散的影响。主要的办法是减小晶体的厚度,使得光脉冲经过晶体时波包尚未来得及扩散。但是过小的厚度也会使得效率下降。

Krylov等[]用掺钛蓝宝石激光(150fs,780nm,0.6mJ,重复频率1KHz)在KDP晶体上产生二次谐波的理论和实验的比较。这一超短光脉冲很易达到几十Gw/cm2的功率密度。在晶体厚度仅为几mm的情况下,可以忽略群速度色散项。这时方程(1)为

                (3)

设起始的基频波时间波形为高斯型,半高全宽是150fs。并在Ⅰ类相位匹配下用数值法求解(3)。计算所得二次谐波发生的转换效率与输入基频光光强的关系以及输出时基频和倍频光的时间波形如图 所示。实验结果由图 给出。由于基频波的功率密度高所以有很高的转换效率但是群速度的不匹配使效率下降。群速度的不匹配时的倍频波比基频波更早到达出口,同时脉宽加长,特别是基频波为低功率密度时脉宽与晶体长度成正比,这不利于超短脉冲的形成。

\subsection{三波混频的其它过程}

二次谐波发生是非线性光学参量混频的最成功的例子。利用它可将激光的优良特性转移到倍频光,生成新的相干光源。例如用半导体二极管激光泵浦的掺Nd3+或掺Yb3+的固体激光器通过倍频获得绿光,具有功耗低,体积小,光束质量高的特性,已成为产品。然而由于优质基频激光波长很是有限,如果要扩大频率变换的范围,就需采用其他参量混频的方法。

晶体中输入两束不同频率的光,由二阶非线性效应,可产生和频波,差频波。 由于它们是三波相互作用,所以可用§2.1 中列出的耦合波方程来讨论。

\subsubsection{和频发生(SFG)}


对和频发生,输入频率为  二束光, 产生 的和频光。由它们的耦合波方程()可见, 一般存在相位不匹配量 ,即每个频率的光的波矢都不等于同频率的二阶非线性极化波的波矢。所以三波在运行中的能量交换有周期性,只有在相干长度上和频光才首次达到最大,下一个相干长度和频光又将能量反回到两个低频光,如此周期重复。所以不能有效的获得和频波。相位匹配,要使能量单向地流向和频光,必须满足和频发生时的相位匹配条件即

.                       (1)

即                                                     (2)

对于三波非共线相互作用,这时相位匹配条件为矢量形式

                                       (3)             
                                                          
从光子角度看, 相位匹配条件表示光子的动量守恒,而 则为光子的能量守恒。

转换效率必须解耦合波方程并由起始条件决定。提高和频发生转换效率的途径与二次谐波发生是相似的(二次谐波发生可以看成 时的和频发生)。利用晶体的光学双折射可以实现上述和频过程的相位匹配,对极性周期反转的光学材料,也可用准相位匹配。和频发生的光源多采用脉冲激光器,如果用cw激光源,则要用谐振腔。

将和频发生与二次谐波发生组合可以产生新频率的相干光,它的目标是获得紫外相干光。因为这个波段的光有广泛的用途,而紫外波段的激光器则极为稀少。XeCl(波长308nm),KrF(249nm),ArF(193nm) 准分子激光器是它们的代表。这些激光器是脉冲输出,脉冲能量大,峰值功率高,平均功率也大,但其体积大,操作不便,参量混频是一种可能的替代。下例是使用单一调Q YAG激光器产生二倍频,三倍频和四倍频光的实验(转换效率分别为50\%,30\%,10\%).目前人们还在研究将分步产生组合频率光的过程通过介质一次完成,这将更有利于参量混频的应用。不过晶体的透明域是实现更短波长的障碍。
         
对100fs 左右的脉冲,它的空间长度为30μm 左右。晶体厚度和所用波长有关,通常约1mm左右。与二次谐波不同,在三波混频中,三个波都有不同的群速度,如何使这三个波在晶体内尽可能重叠是提高转换效率的关键之一。通常要根据各波的群速度的大小,采用光学延迟的技术以提高转换效率。

\subsubsection{差频发生和参量放大(DFG和OPA)}


差频发生是输入频率为  的光( ), 产生 的差频光.。其中高频波 称为泵浦波,低频波 则叫信号波。差频发生的相位匹配条件是 和(1)(2)式相同,但是输入情况不同。根据曼利- 罗关系,当差频波增大时,输入的信号波 也同时增大,而泵浦波 则减弱,反之亦然。故在差频波产生同时,输入波 会得到放大。所以从信号波的观点也是光学参量放大 (OPA) 。 
由三波的耦合波方程,根据适当的边界条件,可得到方程的解。为简单起见,假定泵浦光 近似无衰减,即令 此时只需写出  两个方程,它们是

                                                      (4)

式中令

                             (5)

当两个低频光的起始值 已知,则一般解是

      (6)

式中参数                                             (7)

依照 (它与输入的泵浦光光强I3成正比)和失匹配量 的相对大小,g 可以是实数也可以是虚数。它决定了三个波在传播中的能量转换方式。当进一步假定起始时差频波为零 则可得到差频发生(或OPA)的解为: 

                                  (8)

对应的光强是

                         (9)

可见,当 ,则g 是实数,这时两个波都是单调增。显然它们的能量均来自泵浦波。定义光学参量放大的增益为 则由(9)式可得

                 (10)                      
                    
反之当 ,则g 是纯虚数,令 ,则(9)式成为

                                    (11)

这时 同步地周期振荡,实现与泵浦波周期性地能量交换。

当相位匹配时可得到最佳的参量放大增益。实现相位匹配的方法与二次谐波发生,和频发生类似。在(9)式中令 就得到

                                 (12)                

式中参数 与泵浦光的光强有关。应当说使用cw激光器或常规的调Q激光器,可获得的信号增益是不大的。例如取晶体的有效非线性极化率 ,可见光区的折射率取为1.5,晶体长1cm,如果用氩离子激光器作光源,波长 ,取输出功率10W,光束半径 。可以算出光强是 。由(5)式可算出 ,所以 。如果用调Q脉冲激光器YAG 二倍频 作光源,波长也在  设输出脉冲能量1mJ,脉宽10ns,聚焦光班为 得功率密度 ,由此算得 ,即信号光的增量仅为输入的1.2\%

超快激光器的发展已经使光学参量放大成为实用的变频器件。尽管平均功率不高,超短光脉冲的峰值功率却很高,特别通过啁啾脉冲放大的技术,可使脉冲能量达到mJ,峰值功率达到几十GW,可获得连续调谐的增益。

\subsection{光学参量振荡  (OPO)}
由以上讨论可知,当光源为cw,或调Q激光器时,单程光学参量放大的增益一般很小。为提高转换效率,可将晶体放在谐振腔内,泵浦光和信号光沿腔轴输入。通过腔镜镀膜能使一定频率的光在腔内振荡。如果只使一束低频光振荡,就叫单谐振(SRO)结构,如果使二束低频光都振荡就叫双谐振(DRO) 结构。加谐振腔等于大大延长了参量放大的路程,使得在腔内谐振的光多次往返因而在泵浦光的作用下,其能量不断得到放大,提高了增益,所以可以降低对泵浦功率的要求。  实际上只要泵浦光功率密度超过阈值,不需从外界输入信号光,就能输出二个低频光。 这二束低频光都是从腔内的噪声中生长起来的。这样的器件就叫光学参量振荡器.。在参量振荡器中两个低频光常称为信号光和闲置光。

\subsubsection{振荡阈值}

要能实现参量振荡, 泵浦光必须达到或超过阈值, 阈值条件由腔内谐振光的双程增益等于双程损耗决定。

对单谐振(SRO)结构,设腔对泵浦光和闲置光是透明的,而对信号光则谐振。令两个腔镜对信号光的反射率都是 ,则信号光的振幅反射系数是

                                               (1)

 是镜面反射时的附加相位。由稳定振荡条件得到

 
或者                                         (2)

将实部和虚部分开,得到

                             (3)

其中第一式表示腔纵模的频率条件,第二式与泵浦光强有关,它决定了单谐振腔的振荡阈值条件。将上一节 的公式(5)代入上式就有

                             (4)              

这个阈值与单位长度腔损耗成正比,腔损耗愈大则阈值也愈高。此外选高非线性极化率的晶体可以减小阈值。

对双谐振(DRO)结构,腔对泵浦光是透明的, 对信号光和闲置光都谐振,镜面反射率分别是 。此时在镜面处 都须满足稳定振荡条件。利用前一节的公式(6)并假设 ,便有

              (5)                     

这组关于 的方程要有非零解则系数必须满足

   (6)  

由此得到

        (7)         

式中的第一行表示信号光和闲置光都必须是腔的纵模,但是信号光子和闲置光子又必须满足能量守恒和动量守恒,这些众多的条件使得双谐振的参量振荡器对腔结构和泵浦光源的参数要求很高,输出功率和频率很不容易稳定。但是它的优点是阈值比单谐振结构要低。

由上式第二行考虑到为了有效放大多采用高反射率即 ,这时近似有 ,最后可得双谐振腔的振荡阈值

                   (8)      

比较两种结构的阈值得到

                             (9)      

当取 时,SRO的阈值为DRO阈值的100倍,所以单谐振腔多采用脉冲激光器来泵浦,而双谐振腔可以用连续波激光器泵浦。但是如果将单谐振腔放在激光腔内,或者使泵浦光也在单谐振腔内震荡,则可以用连续波激光器泵浦SRO。

举例来说典型的SRO(泵浦光一次通过腔)泵浦阈值功率是瓦量级,DRO(泵浦光一次通过腔)则为几十毫瓦,SRO(泵浦光也在腔内振荡)阈值功率降为几百毫瓦,DRO(泵浦光也在腔内振荡)则为几毫瓦。

\subsubsection{频率调谐}

光学参量振荡器是一种可以产生很宽的可调谐范围的光源。当泵浦光的频率一定,在参量振荡器内只有一对特定的信号光 闲置光 才能满足条件

                                   (10)          

在共线条件下有

                           (11)                    

所以只要能改变影响折射率 的参数,使得相位匹配关系 仍成立,就可得到新的一对低频光。连续改变折射率 , ,输出光频 也连续改变, 即可以实现频率的连续调谐.

改变折射率的基本方法可以利用晶体的双折射,作角度调谐或温度调谐。除此以外,如果泵浦光源可以倍频,或者具有宽带可连续调谐的特性,则参量振荡的可调谐范围就可以大为扩大。

下面 以负单轴晶体Ⅰ类匹配为例,定量讨论参量振荡器的调谐特性。首先处理角度匹配。设在某个匹配角 , 的关系成立,以折射率表示就有

                          (12)               

现在将晶体光轴相对于通光方向移动一个小角度,使之夹角为 ,这时符合新的相位匹配条件的一组频率将是 ,并有关系

               (13)       

当 很小时, 也是小量,可将折射率在相位匹配值附近展开,

    (14)       

保留一阶小量得到

           (15)     

其中折射率对频率的导数由Sellmeier公式决定,非常光折射率对角度的导数是

         (16)              

(15)式表示当匹配角改变一个小量时,信号光和闲置光频率的改变量。由此可得到新的一组低频光。再以新的匹配角为起点重复以上计算,如此就可得到不同角度时的输出频率。但是上面公式(15)在  (简并)处不适用,这时要从(14)式出发考虑二阶小量,便可得到

                  (17)         

在一定频率的泵浦光作用下,晶体中信号光和闲置光频率随匹配角的变化曲线叫调谐曲线。

 
对于温度调谐可类似讨论。以负单轴晶体Ⅰ类匹配为例,这时在某一温度 下相位匹配可表示为
                                  
(18)

当温度改变一个小量,则低频光的频率改变是

            (19)

当 时,有

                       (20)        

由此也可得到温度调谐曲线。

以上讨论假定泵浦光源是单色的,当光源是宽带可调谐时,OPO的调谐范围还可扩大,如果将光源作参量倍频(二倍频,三倍频或四倍频),则OPO事实上可在紫外到中红外的全波段实现连续可调。随着优质非线性光学晶体以及宽带可调谐激光光源的发展,光学参量振荡器已成为一种重要的相干光源。

用超短光脉冲实现参量振荡的困难之一是保持泵浦脉冲与在谐振腔内振荡的信号脉冲和闲置脉冲每次都要同步地进入晶体,它可以用伺服系统通过反馈信号控制谐振腔的腔长来解决。

OPO输出的线宽见M,Ebrahimzadeh et.al. IEEE J-QE 26(7)(1990)1241

\begin{exercise}
    
   ⒈已知KDP为负单轴晶体, 对Nd :YAG激光( 其平均折射率为n=1.5,有
   效非线性极化率 ,若单激光脉冲能量为 100 m J, 脉冲宽10ns,聚焦光束半
   径为1 mm,求晶体SHG的有效长度.若晶体实长L=2cm, 相位匹配时求转换效率.\\
      ⒉  已知KDP为负单轴晶体,对红宝石激光( 其主折射率为 ,在其倍频光处, ,求Ⅰ类,Ⅱ类相位匹配角.\\
       ⒊ 已知KDP对Nd :YAG激光( 其主折射率为 求Ⅰ类,Ⅱ类相位匹配角.\\
   ⒋  设一 负单轴晶体属于3m晶类,其二阶非线性极化率为 
   求Ⅰ,Ⅱ类相位匹配下的 取何值时, 的绝对值最大?  能不能作温度匹配?\\
      ⒌ 分别推导负﹑正单轴晶体二次谐波发生的容许角宽 和容许温度范围 的公式.\\ 
      ⒍  已知 为强泵浦光,  为弱光,在泵浦光近似无衰减并满足相位匹配条件下,求和频波的光强公式,解释它在空间周期性变化的原因.\\
      ⒎   为负单轴晶体属于3m晶类,在Ⅰ类温度匹配下,泵浦光 ,信号光 ,晶体长 ,   , .求信号光参量放大的增益系数.三种波长的折射率近似都取n=2.3\\
      ⒏ 推导负单轴晶体OPO 角度调谐和温度调谐公式..分Ⅰ类匹配和Ⅱ类匹配两种情形讨论.\\
      9. 参考铁电晶体LiNbO3 的Sellmeier 公式,计算用它的电畴周期性反转实现SHG准相位匹配的反转周期. 已知输入光是Nd:YAG 激光,波长是λ=1.064μm\\
      10. 由上题,计算电畴周期性反转实现OPO准相位匹配的反转周期. 泵浦光同上,要求信号光波长是λ=1.5μm. (提示: 先算出闲置光波长)\\
      11. 证明,对于二次谐波发生 ,准相位匹配时输入光的允许(临界)线宽比同一介质用常规相位匹配的允许线宽要小很多.\\
      12. 以负单轴晶体作二次谐波发生,计算在Ⅰ类匹配时激光的临界线宽 公式,证明此式等价于群速度匹配条件(§2.2.5 (2)式)\\
\end{exercise}

