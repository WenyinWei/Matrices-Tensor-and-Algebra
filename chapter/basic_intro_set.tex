

\subsection{Constructions of sets}

\begin{enumerate}
    \item Given a set $S$. The power set $P(S)$ of $S$ is defined to be the set of all subsets of $S$. \textit{E.g.},
    $S = \{1, 2, 3\}$, $P(S) = \{\emptyset, \{1\}, \{2\}, \{3\}, \{1, 2\}, \{1, 3\}, \{2, 3\}, \{1, 2, 3\}\}$.
    \item  Let I be a non-empty set and for each $i \in I$, one has a set $A_i$ Then one can define
    \begin{itemize}
        \item $\bigcap_{i \in I} A_{i}$, the \textcolor{purple}{\textit{intersection}} of all sets $A_{i}$.
        \item $\bigcup_{i \in I} A_{i}$, the \textcolor{purple}{\textit{union}} of all sets $A_{i}$.
        \item $\prod_{i \in I} A_{i}$, the \textcolor{purple}{\textit{Cartesian product}} of all sets $A_{i}$.
    \end{itemize}
\end{enumerate}

\begin{definition}
    Let $A$ and $B$ be two non-empty sets. A function $f : A \rightarrow B$ is a rule that assigns to every
    element $a$ of $A$ a unique element $b$ of $B$. Equivalently, the function $f$ can be described
    as a subset $\Gamma \subset A \times B$ satisfying the following conditions:
    \begin{enumerate}
        \item $\forall a \in A$, $\in b \in B$ such that $(a, b) \in \Gamma$;
        \item if $(a, b)$ and $(a, b^{\prime})$ belong to $\Gamma$, then $b = b$
    \end{enumerate}
\end{definition}


Given a function $f : A \rightarrow B$, one can talk about whether $f$ is injective, surjective,
bijective, inverse function of $f$ (if bijective), images and preimages.

Given two functions $f : A \rightarrow B$ and $g : B \rightarrow C$, one can define the composition function
$g \circ f : A \rightarrow C$ that maps a to $g(f(a))$.

\begin{lemma}
    Any set $S$ is not bijective to its power set $P(S)$.
\end{lemma}

\subsection{Equivalence relations}
\begin{definition}
    Let $S$ be a non-empty set. A \textcolor{purple}{\textit{relation $\sim$ on $S$}} is defined by a subset $R \subset  S \times  S$ in
the following sense: we say $a \sim b$ iff $(a, b) \in R$. A set $S$ equipped with a relation $\sim$ is
denoted by $(S, \sim)$.
\end{definition}

\begin{definition}
Given $(S, \sim)$, the relation $\sim$ is said to be an \textcolor{purple}{\textit{equivalence relation}} if the following conditions
hold:
\begin{enumerate}
    \item $\forall x \in S$, $x \sim x$ (reflexive);
    \item if $x \sim y$, then $y \sim x$ (symmetric);
    \item if $x \sim y$ and $y \sim z$, then $x \sim z$ (transitive).
\end{enumerate}

\end{definition}




If $S$ is equipped with an equivalence relation $\sim$ and $x, y \in S$, we say x is equivalent to
$y$ if $x \sim y$. The subset $[x] := \{y \in S| y \sim x\}$ of $S$ consisting of all elements equivalent
to $x$ is called the equivalence class of $x$. The set of all equivalence classes is denoted by
$S/ \sim$.

Equivalence relations occur everywhere, \textit{e.g.}, one can define an equivalence relation on
$Z$ by saying that $m \sim n$ if $m - n$ is divisible by $2$, in this case there are two equivalence
classes: the class of even numbers and the class of odd numbers. Another example, let
$S$ be the set of Chinese people and we define two Chinese $x$ and $y$ to be equivalent iff
$x$ and $y$ have the same surname, in this case the equivalence classes correspond to the
surnames of Chinese.

\begin{lemma}
A fundamental FACT about equivalence relation $(S, \sim): \forall x, y \in S$, either $[x] = [y]$ or
$[x] \bigcap [y] = \emptyset$. This fact implies that $S$ is the disjoint union of (or partitioned by) the
equivalence classes of S. Equivalently, one can defines an equivalence relation $\sim$ on $S$ by
partitioning $S$ into disjoint non-empty subsets and saying that $ x \sim y$ iff $x$ and $y$ belong
to the same partition.
\end{lemma}

Given $S$ with equivalence relation $\sim$, one has a natural (surjective) function $\pi: S \rightarrow S/ \sim$
mapping $x \in S$ to $[x] \in S/ \sim$.

\subsection{Partial Order Set and \textit{Zorn's Lemma}}

\begin{definition}
    Let $S$ be a set equipped with a relation $\leq$. We say that $(S, \leq)$ is a \textcolor{purple}{\textit{partial order set}} if
    the following conditions hold:
    \begin{enumerate}
        \item $\forall x \in S$, $x \leq x$;
        \item if $x \leq  y$ and $y \leq  x$, then $x = y$;
        \item if $x \leq  y$ and $y \leq  z$, then $x \leq z$.
    \end{enumerate}

\end{definition}

\begin{definition}
A partial order set $(S, \leq )$ is said to be a \textcolor{purple}{\textit{total order set}} if $\forall x, y \in S$, either $x \leq  y$ or
    $y \leq  x$ holds.
\end{definition}

\begin{example}
    Given a set $S$, then $(P(S), \subset )$ is a partial order set (but not a total order set in general).
\end{example}

\begin{definition}
    \textbf{Chain, Upper Bound and Maximal Element}\\
    \begin{enumerate}
        \item Given a partial order set $(S, \leq )$, a subset $C \subset  S$ is called a \textcolor{purple}{\textit{chain}} if $\forall x, y \in C$ either
        $x \leq  y $or $y \leq  x$ holds.
        
        \item Given a partial order set $(S, \leq )$ and a subset $T \subset  S$, an element $x \in S$ is called an \textcolor{purple}{\textit{upper
        bound}} of $T$ if$ \forall y \in T$, $y \leq  x$ holds.
        
        \item Given a partial order set $(S, \leq )$, an element $x \in S$ is called a \textcolor{purple}{\textit{maximal element}} if there
        does not exist $y \in S $ such that $x \leq  y$ and $y = x$. Remark: $(S, \leq )$ can have multiple
        maximal elements.
    \end{enumerate}
\end{definition}


\begin{lemma}
\textcolor{purple}{\textbf{Zorn’s lemma}}\\
 Let $(S, \leq )$ be a partial order set. If every chain $C \subset  S$ has an upper
bound in $S$, then $S$ has a maximal element.

One can prove by \textit{Zorn’s lemma} that every non-zero vector space has a basis.
\end{lemma}