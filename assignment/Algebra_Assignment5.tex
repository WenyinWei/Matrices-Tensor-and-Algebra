\subsection{Assignment 5}

\begin{exercise}
    1.
\end{exercise}

\begin{proof}
    The problem is equivalent to that the smallest non-cyclic simple group has order 60.

    All the numbers to be considered are listed as follow.\\

    \begin{tabular}{llllllllll}
    \toprule
    1  & 2  & 3  & 4  & 5  & 6  & 7  & 8  & 9  & 10 \\ 
    11 & 12 & 13 & 14 & 15 & 16 & 17 & 18 & 19 & 20 \\
    21 & 22 & 23 & 24 & 25 & 26 & 27 & 28 & 29 & 30 \\
    31 & 32 & 33 & 34 & 35 & 36 & 37 & 38 & 39 & 40 \\
    41 & 42 & 43 & 44 & 45 & 46 & 47 & 48 & 49 & 50 \\
    51 & 52 & 53 & 54 & 55 & 56 & 57 & 58 & 59 &    \\ \bottomrule
    \end{tabular}\\

    Firstly, we rule out all prime order groups, which are definitely cyclic groups and simple, since they even have no non-trivial subgroups, let along the normal subgroups.\\

    \begin{tabular}{llllllllll}
        \toprule
          &   &   & 4  &   & 6  &   & 8  & 9  & 10 \\ 
         & 12 &  & 14 & 15 & 16 &  & 18 &  & 20 \\
        21 & 22 &  & 24 & 25 & 26 & 27 & 28 &  & 30 \\
         & 32 & 33 & 34 & 35 & 36 &  & 38 & 39 & 40 \\
         & 42 &  & 44 & 45 & 46 &  & 48 & 49 & 50 \\
        51 & 52 &  & 54 & 55 & 56 & 57 & 58 &  &    \\ \bottomrule
    \end{tabular}\\

    Note that if a group has a nontrivial center (denoted $C(G)$), it cannot be simple. In particular, the center is a normal subgroup itself. On the other hand, if the center is all of $G$ (i.e. $G$ is abelian) then any subgroup of $G$ is automatically a normal subgroup, and since all groups of non-prime order have nontrivial subgroups, they also cannot be simple. To exploit this, we will utilize the following theorem, that every group of prime power order is not simple:


    \begin{theorem}
    Every group $G$ of order $p^n$ (\textit{i.e.}, $p$-group) has a non-trivial center. (\textit{Theorem 7-3} in our book.)
    \end{theorem}

    In particular, this rules out groups of order 4, 8, 9, 16, …, leaving us with the following table:\\

    \begin{tabular}{llllllllll}
        \toprule
          &   &   &   &   & 6  &   &   &   & 10 \\ 
         & 12 &  & 14 & 15 &  &  & 18 &  & 20 \\
        21 & 22 &  & 24 &  & 26 &  & 28 &  & 30 \\
         &  & 33 & 34 & 35 & 36 &  & 38 & 39 & 40 \\
         & 42 &  & 44 & 45 & 46 &  & 48 &  & 50 \\
        51 & 52 &  & 54 & 55 & 56 & 57 & 58 &  &    \\ \bottomrule
    \end{tabular}\\

    Next, specifically, we need to determine whether every group of order $n$ above has a normal subgroup. Enter the Sylow theorems:

% Recall that a Sylow $p$-subgroup is a subgroup of a group $G$ of order $p^r$ where $p$ is prime, such that $p^{r+1}$ does not divide the order of $G$. We will call this maximal prime power order. 
By utility of the \textit{Three Sylow Theorems}, we ruled out a series of numbers.
\begin{itemize}
    \item The \nth{1} Sylow theorem states that this subgroup exists.
    \item The \nth{2} Sylow theorem states that any two Sylow $p$-subgroups are conjugate, which implies that if there is any Sylow $p$-subgroup is conjugate with itself, then it must be normal. 
    \item Finally, the \nth{3} Sylow theorem helps us count the number of Sylow $p$-subgroups of any group. Specifically, it states that if the order of $G$ is $p^{r}m$ where $p$ does not divide $m$, then the number of Sylow $p$-subgroups divides $m$ and is congruent to $1$ modulo $p$.
\end{itemize}


Our first application will be the following theorem:

\begin{theorem}
    If $G$ has order$ mp^r$ where $p$ is prime and $1 < m < p$, then $G$ is not simple.
\end{theorem}


\begin{proof}
    There is at least one Sylow $p$-subgroup of G, and the total number of such groups is $1$ mod $p$ and divides $m$. Since $m < p$, there can only be one such subgroup. Hence, the unique Sylow $p$-subgroup is normal, and $G$ is not simple.
\end{proof}

This rules out the numbers 6, 10, 14, 15, 18, …, to leave us with the now sparse list:\\

\begin{tabular}{llllllllll}
    \toprule
      &   &   &   &   &   &   &   &   &  \\ 
     & 12 &  &  &  &  &  &  &  &  \\
     &  &  & 24 &  &  &  &  &  & 30 \\
     &  &  &  &  & 36 &  &  &  & 40 \\
     &  &  &  & 45 &  &  & 48 &  &  \\
     &  &  &  &  & 56 &  &  &  &    \\ \bottomrule
\end{tabular}\\

There are just a few numbers left.

\begin{enumerate}[<i>]
    \item For $|G| = 40$, there is a unique Sylow $5$-subgroup (the only divisor of 8 which is $1$ mod $5$ is $1$), and the same argument holds for $|G| = 45$, with $9$ in the place of $8$.
    \item For $|G| = 30$, we may count both the number of Sylow 3-subgroups and the Sylow 5-subgroups. For the former, there are either 1 or 10 (the only divisors of 10 which are 1 mod 3), and for the latter there are either 1 or 6 (only divisors of 6 which are 1 mod 5). But certainly there can’t be 6 subgroups of order 5 and 10 of order 3, since these subgroups cannot intersect, and this would account for far too many elements. Hence, there is either a unique Sylow 5-subgroup or a unique Sylow 2-subgroup. Either way, G is normal.
    \item Similarly, for $|G| = 56$, we have the number of Sylow 7-subgroups is either 1 or 8, and the number of Sylow 2-subgroups is either 1 or 7. If neither are 1, we have 6 distinct generators of each Sylow 7-subgroup, giving 48 distinct elements. We further have at least one Sylow 2-subgroup H of order 8, giving a total of 56 elements, but since there are more Sylow 2-subgroups we can pick an element in one which is not H, and find ourselves with more than |G| elements.
\end{enumerate}

Now only the multiples of twelve are survived.\\

\begin{tabular}{llllllllll}
    \toprule
      &   &   &   &   &   &   &   &   &  \\ 
     & 12 &  &  &  &  &  &  &  &  \\
     &  &  & 24 &  &  &  &  &  &  \\
     &  &  &  &  & 36 &  &  &  &  \\
     &  &  &  &  &  &  & 48 &  &  \\
     &  &  &  &  &  &  &  &  &    \\ \bottomrule
\end{tabular}\\

If $|G| = 36$ then we have either 1, 3, or 9 Sylow $2$-subgroups, and 1 or 4 Sylow $3$-subgroups, but it is again clear by counting elements that there cannot be both 9 of the former and 4 of the latter, so either G is already simple, or there are at most 3 Sylow 2-subgroups, and we may use the following result to prove all groups with these orders are simple.

Suppose $G$ is a group with any of these remaining orders, and let $N_2$ be the number of Sylow 2-subgroups. The Sylow theorems (and our above investigation of $|G| = 30$) show that $N_2 = 3$ or else G is already simple. Now let G act on the set of Sylow 2-subgroups by conjugation. This is a homomorphism $G \to S_3$, but since $|G| \geq 12 > 6 = |S_3|$, this map cannot be injective. In other words, its kernel is a normal subgroup of G. So G is not simple.

As we have just proved, there are no non-cyclic simple groups of order less than $60$. On the other hand, there is such a group of order 60: $A_5$ (famously discovered by Évariste Galois). It is another theorem altogether that this is the only simple group of order 60.

To be honest, part of the proof is based on \href{https://jeremykun.com/2011/10/08/the-smallest-non-cyclic-simple-group-has-order-60/}{The Smallest Non-Cyclic Simple Group has Order 60}.

\end{proof}


\begin{exercise}
    2.
\end{exercise}

\begin{proof}   
    Not yet finished.
\end{proof}
