\subsection{Tensor Basics, Nonlinear Optics Appendix}

\begin{exercise}
新坐标系为老坐标系绕 $x3$ 轴转 $\phi$ 角而成. 求二阶张量 $T_{ij}$ 在新、老坐标系中的变换关系.

\end{exercise}

\begin{proof}
    To acquire the transformation relation of dyadics under rotation angle $\phi$ around the $x3$ axis. Only the transformation parameters $\alpha_{ij}$ are needed. 
    
    $$
    \alpha_{i j}=\hat{e}_{i}^{\prime} \cdot \hat{e}_{j}=\cos \left(i^{\prime}, j\right)
    $$

    $$
    \alpha_{1 1}=\cos (1^{\prime}, 1)=\cos (\phi) = \alpha_{2 2}, \alpha_{3 3} =1
    $$
    
    $$
    \alpha_{1 2}=\cos (1^{\prime}, 2)=\cos (\pi/2-\phi) = \sin (\phi)
    $$

    $$
    \alpha_{2 1}=\cos (2^{\prime}, 1)=\cos (\pi/2+\phi) = -\sin (\phi)
    $$

    $$
    \alpha =
    \left ( 
        \begin{matrix}
        \cos (\phi) & \sin (\phi) & 0\\
        -\sin (\phi) & \cos (\phi) & 0\\
        0 & 0 & 1 \\
        \end{matrix}
    \right )
    $$

    To do the dyadics transform, one need the following formula,
    $$
    T_{i j}^{\prime}=\alpha_{i k} \alpha_{j l} T_{k l}
    $$.
\end{proof}

\begin{exercise}
    证明,任意二阶张量 $T_{ij}$ 皆可表示为二阶对称张量与二阶反对称张量之和.
\end{exercise}

\begin{proof}
    The symmetric- and antisymmetric two-order tensor can be constructed by the following formula.

    $$ S = (T + T^{T})/2 $$
    $$ A = (T - T^{T})/2 $$
    $$ T = S+A $$

    We use matrix symbol here to show the kernel consisely.

\end{proof}

\begin{exercise}
    证明,任意二阶对称张量 $T_{ij}$ 可分解为一个与张量 $\delta_{ij}$ 成比例的对称张量及一个跡为零的对称张量之和.

    
    To prove that such a decomposition is possible, 

    $$ T = kI + S$$,
    
    where $T$ is a symmetric tensor while $S$ is even a symmetric one with trace zero.
\end{exercise}

\begin{proof}

    We know from the symmetric characteristc that the decomposition only concerns the elements on the diagonal line, $T_{11},T_{22},T_{33}$.

    $$S_{ii}=T_{ii}-k \text{, respectively.}$$

    To make $S_{11}+S_{22}+S_{33}=0$, let $k=(T_{11}+T_{22}+T_{33})/3$.

    We use matrix symbol here to show the kernel consisely.
    
\end{proof}