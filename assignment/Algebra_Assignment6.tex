\subsection{Assignment 6}

\begin{exercise}
    1.
\end{exercise}

\begin{proof}
    \begin{enumerate}[(i)]
        \item $S_4$:
        $$
        \{e\} \triangleleft \{e,(12)(34)\}  \triangleleft\{e,(12)(34),(13)(24),(14)(23)\} \triangleleft A_{4} \triangleleft S_{4}
        $$
        $$
        \{e\} \triangleleft \{e,(13)(24)\}  \triangleleft\{e,(12)(34),(13)(24),(14)(23)\} \triangleleft A_{4} \triangleleft S_{4}
        $$
        $$
        \{e\} \triangleleft \{e,(14)(23)\}  \triangleleft\{e,(12)(34),(13)(24),(14)(23)\} \triangleleft A_{4} \triangleleft S_{4}
        $$
        with composition factors $\{\bbZ_2, \bbZ_2, \bbZ_3, \bbZ_2\}$ up to isomorphism.
        \item $S_5$:
        $$
        \{e\} \triangleleft A_{5} \triangleleft S_{5}
        $$
        with composition factors $\{A_5, \bbZ_2\}$ up to isomorphism
        .
    \end{enumerate}
\end{proof}

\begin{exercise}
    2.
\end{exercise}

\begin{proof}
    $$
    \{e\} \oplus \{e\} \oplus \{e\} \triangleleft 
    \begin{Bmatrix}
    \bbZ_p \oplus \{e\} \oplus \{e\}     \\ 
    \{e\} \oplus \bbZ_p \oplus \{e\}   
    \end{Bmatrix}\triangleleft 
    \bbZ_p \oplus \bbZ_p \oplus \{e\}\triangleleft 
    \bbZ_p \oplus \bbZ_p \oplus \bbZ_p
    $$
    
    $$
    \{e\} \oplus \{e\} \oplus \{e\} \triangleleft 
    \begin{Bmatrix}
    \bbZ_p \oplus \{e\} \oplus \{e\}     \\ 
    \{e\} \oplus \{e\} \oplus \bbZ_p   
    \end{Bmatrix}\triangleleft 
    \bbZ_p \oplus \{e\} \oplus \bbZ_p\triangleleft 
    \bbZ_p \oplus \bbZ_p \oplus \bbZ_p
    $$

    
    $$
    \{e\} \oplus \{e\} \oplus \{e\} \triangleleft 
    \begin{Bmatrix}
    \{e\} \oplus \{e\} \oplus \bbZ_p     \\ 
    \{e\} \oplus \bbZ_p \oplus \{e\}   
    \end{Bmatrix}\triangleleft 
    \{e\} \oplus \bbZ_p \oplus \bbZ_p\triangleleft 
    \bbZ_p \oplus \bbZ_p \oplus \bbZ_p
    $$

    Totally 6 composition series, but they are in fact the same up to isomorphism.
\end{proof}

\begin{exercise}
    3.
\end{exercise}

\begin{proof}
    According to Prof. \textit{Hui}'s note of \textsc{Group Theory 10}, 1.2 Solvable groups (5) Corollary (i), if $G$ is solvable, then subgroups of $G$ must be solvable.

    Due to the fact that $H$ is simple, $H$ only contains a unique normal series that $\{e\}\triangleleft H$. Furthermore, $H$ is non-abelian while the only composition factor in the series is $H$ itself, being non-abelian. This contradicts with the corollary that subgroups of $G$ should be also solvable.
\end{proof}

\begin{exercise}
    4.
\end{exercise}

\begin{proof}

Let $\mathrm{GL}_n(\bbF)$ be the $n$ by $n$ general linear group over $\bbF$,\\ $\mathrm{B}_n(\bbF)$ be the subgroup of $\mathrm{GL}_n(\bbF)$ consisting of all upper triangular matrices,\\ $\mathrm{U}_n(\bbF)$ be the subgroup of $\mathrm{B}_n(\bbF)$ whose diagonal elements are all $1$.\\


$$
\left[\begin{array}{lll|lll}{1} & {x} & {y} & {1} & {0} & {0} \\ {0} & {1} & {z} & {0} & {1} & {0} \\ {0} & {0} & {1} & {0} & {0} & {1}\end{array}\right] \stackrel{R_{1}-x R_{2}}{\longrightarrow}
\left[
    \begin{array}{ccc|ccc}
        {1} & {0} & {y-x z} & {1} & {-x} & {0} \\
        {0} & {1} & {z} & {0} & {1} & {0}  \\
        {0} & {0} & {1} & {0} & {0} & {1} 
    \end{array}
\right]
$$

$$
\stackrel{\begin{array}{c}{R_{1}-(y-x z) R_{3}} \\ {R_{2}-z R_{3}}\end{array} }{\longrightarrow}
\left[\begin{array}{ccc|ccc}
{1} & {0} & {0} & {1} & {-x} & {x z-y} \\ 
{0} & {1} & {0} & {0} & {1} & {-z} \\
{0} & {0} & {1} & {0} & {0} & {1}
\end{array}\right]
$$



If $A$ is a $k \times k$ matrix, let the entries $a_{i,i+l}$ of $A$ belong to the  $l$-th diagonal of $A$. In particular, the $0$-th diagonal of $A$ is its actual diagonal.
\begin{lemma}
    If $A,B\in \mathrm{U}_n(\bbF)$ and $C=[A,B]$ then the first diagonal of $C$ is 0. More in general, if the first $l$ diagonals (except the $0$-th) of $A$ are $0$ and the first $m$ diagonals (except the $0$-th) of $B$ are $0$, then the first $l+m+1$ diagonals of $[A,B]$ are 0.
\end{lemma}
\begin{proof}
    Here is a simplified proof for the $n=3$ case while $l=m=0$.
    Suppose $A,B\in \mathrm{U}_n(\bbF)$,

    \begin{multline}
        A^{-1}B^{-1}AB\\
        =\begin{bmatrix}
            1 & -x_1 & x_1z_1-y_1\\
            {} & 1 & -z_1 \\
            {} & {} & 1  
          \end{bmatrix}
          \begin{bmatrix}
              1 & -x_2 & x_2z_2-y_2\\
              {} & 1 & -z_2 \\
              {} & {} & 1  
            \end{bmatrix}
            \begin{bmatrix}
              1 & x_1 & y_1\\
              {} & 1 & z_1 \\
              {} & {} & 1  
            \end{bmatrix}
            \begin{bmatrix}
              1 & x_2 & y_2\\
              {} & 1 & z_2 \\
              {} & {} & 1  
            \end{bmatrix}\\
            =
            \begin{bmatrix}
                1 & -x_1-x_2 & x_1z_1+x_2z_2-y_1-y_2+x_1z_2\\
                {0} & 1 & -z_1-z_2 \\
                {0} & {0} & 1  
              \end{bmatrix}
              \begin{bmatrix}
                1 & x_1 & y_1\\
                {0} & 1 & z_1 \\
                {0} & {0} & 1  
              \end{bmatrix}\\
            = \begin{bmatrix}
                1 & {0} & x_1z_2-x_2z_1\\
                {0} & 1 & {0} \\
                {0} & {0} & 1  
              \end{bmatrix}
    \end{multline}
    Similarly for the cases of $l=1,m=0$ and $l=0,m=1$. 

\end{proof}

As long as one does twice commutator operations, the result must be the $e$. That means $\mathrm{U}_n(\bbF)^{(2)}=\{e\}$, implying $\mathrm{U}_n(\bbF)$ is solvable. In addition, $\mathrm{B}_n(\bbF)/\mathrm{U}_n(\bbF)$ is indeed isomorphic to the group of diagonal matrices. It suffices to prove that $\mathrm{B}_n(\bbF)$ is solvable.

\end{proof}

\begin{exercise}
    5.
\end{exercise}

\begin{proof}
    $A_4$ is pretty well known because of its strange number of elements of each order (order of elements divides order of the group): \\
% Please add the following required packages to your document preamble:


    \begin{tabular}{rlllll}
    \toprule
    Order              & 1 & 2 & 3 & 4 & 6 \\ \midrule
    Number of Elements & 1 & 3 & 8 & 0 & 0 \\ \bottomrule
    \end{tabular}\\\\
    
    However, $S_3 \oplus \bbZ_2$ have elements of order 4 or 6, say, $\{(123), 1\}$ with order 6. 
\end{proof}

\begin{exercise}
    6.
\end{exercise}

\begin{proof}
    One may come up with the rotation matrix which is appropriate to construct the isomorphism.
    
    $$(1)(2)(3)\sim \begin{bmatrix}
        1 & 0\\
        0 & 1
    \end{bmatrix}$$,
    $$(123)\sim \begin{bmatrix}
        \cos \frac{2\pi}{3} & -\sin \frac{2\pi}{3}\\
        \sin \frac{2\pi}{3} & \cos \frac{2\pi}{3}
    \end{bmatrix}$$,
    $$(132)\sim \begin{bmatrix}
        \cos \frac{4\pi}{3} & -\sin \frac{4\pi}{3}\\
        \sin \frac{4\pi}{3} & \cos \frac{4\pi}{3}
    \end{bmatrix}.$$

    It is clear that we even don't need a $\mathrm{GL}_2(\bbC)$ condition, a $\mathrm{GL}_2(\bbR)$ being enough.
\end{proof}

\begin{exercise}
    7. Can $A_4$ be a subgroup of $\mathrm{GL}_2(\bbC)$?
\end{exercise}

\begin{proof}
    There is a theorem indicated by \textit{Klein} that 
    \begin{theorem}
        A finite subgroup of $\mathrm{PGL}_2(\bbC)$ is isomorphic to one of the following polyhedral groups:
\begin{itemize}
    \item a cyclic group $C_n$;
    \item a dihedral group $D_n$ of order $2n$, $n \geq2$;
    \item the tetrahedral group $A_4$ of order 12;
    \item the octahedral group $S_4$ of order 24;
    \item the icosahedral group $A_5$ of order 60.
\end{itemize}
    \end{theorem}
    In addition, the projective general linear group $\mathrm{PGL}_2(\bbC)\leq \mathrm{GL}_2(\bbC)$. It suffices to say that $A_4 \leq \mathrm{GL}_2(\bbC)$
\end{proof}