\subsection{Assignment 6}

\begin{exercise}
    1.
\end{exercise}

\begin{proof}
    \begin{enumerate}[(i)]
        \item $S_4$:
        $$
        \{e\} \triangleleft \{e,(12)(34)\}  \triangleleft\{e,(12)(34),(13)(24),(14)(23)\} \triangleleft A_{4} \triangleleft S_{4}
        $$
        $$
        \{e\} \triangleleft \{e,(13)(24)\}  \triangleleft\{e,(12)(34),(13)(24),(14)(23)\} \triangleleft A_{4} \triangleleft S_{4}
        $$
        $$
        \{e\} \triangleleft \{e,(14)(23)\}  \triangleleft\{e,(12)(34),(13)(24),(14)(23)\} \triangleleft A_{4} \triangleleft S_{4}
        $$
        with composition factors $\{\bbZ_2, \bbZ_2, \bbZ_3, \bbZ_2\}$ up to isomorphism.
        \item $S_5$:
        $$
        \{e\} \triangleleft A_{5} \triangleleft S_{5}
        $$
        with composition factors $\{A_5, \bbZ_2\}$ up to isomorphism
        .
    \end{enumerate}
\end{proof}

\begin{exercise}
    2.
\end{exercise}

\begin{proof}
    $$
    \{e\} \oplus \{e\} \oplus \{e\} \triangleleft 
    \begin{Bmatrix}
    \bbZ_p \oplus \{e\} \oplus \{e\}     \\ 
    \{e\} \oplus \bbZ_p \oplus \{e\}   
    \end{Bmatrix}\triangleleft 
    \bbZ_p \oplus \bbZ_p \oplus \{e\}\triangleleft 
    \bbZ_p \oplus \bbZ_p \oplus \bbZ_p
    $$
    
    $$
    \{e\} \oplus \{e\} \oplus \{e\} \triangleleft 
    \begin{Bmatrix}
    \bbZ_p \oplus \{e\} \oplus \{e\}     \\ 
    \{e\} \oplus \{e\} \oplus \bbZ_p   
    \end{Bmatrix}\triangleleft 
    \bbZ_p \oplus \{e\} \oplus \bbZ_p\triangleleft 
    \bbZ_p \oplus \bbZ_p \oplus \bbZ_p
    $$

    
    $$
    \{e\} \oplus \{e\} \oplus \{e\} \triangleleft 
    \begin{Bmatrix}
    \{e\} \oplus \{e\} \oplus \bbZ_p     \\ 
    \{e\} \oplus \bbZ_p \oplus \{e\}   
    \end{Bmatrix}\triangleleft 
    \{e\} \oplus \bbZ_p \oplus \bbZ_p\triangleleft 
    \bbZ_p \oplus \bbZ_p \oplus \bbZ_p
    $$

    Totally 6 composition series, but they are in fact the same up to isomorphism.
\end{proof}

\begin{exercise}
    3.
\end{exercise}

\begin{proof}
    According to Prof. \textit{Hui}'s note of \textsc{Group Theory 10}, 1.2 Solvable groups (5) Corollary (i), if $G$ is solvable, then subgroups of $G$ must be solvable.

    Due to the fact that $H$ is simple, $H$ only contains a unique normal series that $\{e\}\triangleleft H$. Furthermore, $H$ is non-abelian while the only composition factor in the series is $H$ itself, being non-abelian. This contradicts with the corollary that subgroups of $G$ should be also solvable.
\end{proof}

\begin{exercise}
    4.
\end{exercise}

\begin{proof}
    
\end{proof}

\begin{exercise}
    5.
\end{exercise}

\begin{proof}
    
\end{proof}

\begin{exercise}
    6.
\end{exercise}

\begin{proof}
    
\end{proof}