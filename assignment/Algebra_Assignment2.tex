\subsection{Assignment 2}

\begin{exercise}
1. Let $G$ be a group. Prove that the order of $g$ is equal to the order of $g^{-1}$ for all $g\in G$.
\end{exercise}

\begin{proof}
By definition, $ord(g)=min\{n\in\bbN_+|g^n=e\}$, except for the $ord(g)=\infty$ case. For the $n\in\bbN, g^n=e$, multiplied by $g^{-n}$ at the left or right, it is easy to see that $g^{-n}=e$. One can say that $\{n\in\bbN|g^n=e\}=\{n\in\bbN|g^{-n}=e\}$. Therefore the proposition is evidently right.

For the trivial case, $min\{n\in\bbN|g^n=e\}=\infty$, $g^n$ is never the inverse of $g$ for all $n\in \bbN_+$. We just need to prove that $g^{-n}$ is never the inverse of $g^{-1}$, which can be acquired by $g^{-(n-1)}g^n\neq g^{-(n-1)}g^{-1}, \forall n\in\bbN_+$, that is, $g\neq g^{-n}, \forall n\in\bbN_+$.
\end{proof}

\begin{exercise}
    Let $G$ be a group with group law $\bullet$, that is, the product of $g$ and h is $g \bullet h$ for all $g, h \in G$.
    Define a new binary operation $* : G \times G \rightarrow $G as $g * h := h \bullet g$. Prove that this defines a group structure on G (called the \textit{opposite group}).
\end{exercise}

\begin{proof}
    \begin{enumerate}
        \item (Closed), $\forall g,h \in G, g*h=h\bullet g\in G$.
        \item (Identity), evidently, the identity element $e_{\bullet}$ of $\bullet$ operation is also the itentity element $e_{*}$ of $*$ operation.
        \item (Inverse), we $g^{-1}_{\bullet}$ denotes the inverse of $\bullet$ operation. $g*g^{-1}_{\bullet}=g^{-1}_{\bullet} \bullet g=e$. $g^{-1}_{\bullet}=g^{-1}_{*}$
    \end{enumerate}
\end{proof}

\begin{exercise}
    Let $G$ be a group with subgroups $K$ and $H$ such that $K \leq H \leq G$. Prove the following statements.
    \begin{enumerate}
        \item  The index $[G : K]$ is finite if and only if $[G : H]$ and $[H : K]$ are finite.
        \item If $[G : K]$ is finite, then $[G : K] = [G : H] \times [H : K]$.

    \end{enumerate}
\end{exercise}

\begin{proof}
    \begin{enumerate}
        \item 
        
        Let $H=\bigcup Kb_i$ be $H$ right coset decomposition by $K$, $G=\bigcup Ha_j$ be $G$ right coset decomposition by $H$. Naturally, $G=\bigcup Kb_ia_j$ is $G$ right coset decomposition by $K$.
        \item 
        \textit{Lagrange theorem} is available, $|G|=|K|\cdot [G:K]$. $H \subset G$, then $|G|=|H|\cdot [G:H]$ and $|H|=|K|\cdot [H:K]$ both hold. Therefore $[G : K] = [G : H] \times [H : K]$ holds.
    \end{enumerate}
\end{proof}

\begin{exercise}
    Let $G$ be a group with subgroups $K$ and $H$. Prove the following statements
    \begin{enumerate}
        \item  If $K$ is normal in $G$, then $H \bigcap K$ is normal in $H$.
        \item Give an example of $K \leq H \leq G$ such that $K$ is normal in $H$ and $H$ is normal in $G$ but $K$ is not normal in $G$.
    \end{enumerate}
\end{exercise}

\begin{proof}
    \begin{enumerate}
        \item 
        $H\bigcap K\subset K$, which means that $$\forall k\in H\bigcap K, kg=gk, \forall g\in G.$$ $H\subset G$, so $$\forall k\in H\bigcap K, kg=gk, \forall g\in H$$ also holds. 
        To prove that $H\bigcap K\triangleleft H$, $ kg=gk, \forall k\in H\bigcap K, \forall g \in H$ is enough.
        \item 
        A famous smallest examples is presented here. Let $G$ be the dihedral group of order eight and $H$ a cyclic group of order two, with $H_1$ and $H_2$ being subgroups of order two generated by reflections. Usual definition is as follow:
        $$
        \begin{array}{l}{G=\left\langle a, x | a^{4}=x^{2}=e, x a x=a^{-1}\right\rangle} \\ { H_{1}=\langle x\rangle, \quad H_{2}=\left\langle a^{2} x\right\rangle, \quad K=\left\langle x, a^{2}\right\rangle}\end{array}
        $$

        $H_1,H_2\triangleleft K$, $K\triangleleft G$. However, $H_1\triangleleft G$ and $H_2\triangleleft G$ do not hold.
    \end{enumerate}
\end{proof}

\begin{exercise}
    Let $G$ be a finite subgroup of $\mathrm{GL}_n(\bbC)$ (the group of $n$ by $n$ invertible complex matrices).
Prove the following statements.
\begin{enumerate}
    \item Every $g \in G$ is diagonalizable.
    \item  If $G$ is abelian, then all elements in $G$ can be simultaneously diagonalized.
\end{enumerate}
\end{exercise}

\begin{proof}
    \begin{enumerate}
        \item $G$ is closed and finite under matrix multiplication. Firstly the absolute values of eigenvalues $|\lambda_i|$ have to be 1, otherwise $g\in G$ has eigenvalue $|\lambda_i|\neq 1$, then $g^n\in G$ has eigenvalue $|\lambda_i|^n\neq 1$. Then there appear infinite elements. It is contradictory to the finite group hypothesis. Since the elements $g\in G \subset \mathrm{GL}_n(\bbC)$, the proposition is equivalent with that non-singular complex matrix must has $n$ distinct eigenvectors.
        \item $G$ is abelian, then $\forall g_1, g_2 \in G, g_1 g_2 =g_2 g_1$
    \end{enumerate}
\end{proof}

\begin{exercise}
    Let $G$ be a finite group of order $pq$, where $p$ and $q$ are distinct prime numbers. Prove the
following statements.
\begin{enumerate}
    \item The center $C(G)$ is either $\{e\}$ or $G$.
    \item  There exist elements $g$ and $h$ of $G$ such that $ord(g) = p$ and $ord(h) = q$ (hint:
    consider the equivalence relation on $G: g_1 \sim g_2$ if and only if $h^{-1} g_{1} h=g_{2}$ for some
    $h \in G$).
\end{enumerate}
\end{exercise}

\begin{proof}

    \begin{enumerate}
        \item When the center of $G$ is $G$, $G$ is abelian. To prove  the proposition, only need to confirm that, when $G$ is not abelian, the center of $G$ is necessarily $\{e\}$. That means $\forall g,h \in G, gh \neq hg$ except for the case $g=e$. 
        
        We know from the \textit{Lagrange's theorem} that the finite group $G$ can only have subgroups of specified orders, $1, p, q, pq$. The trivial subgroups are $\{e\}$ and $G$. So we can focus on the subgroups of orders $p$ and $q$.
    \end{enumerate}

    A subgroup decides a right coset decomposition, vice versa. By virtue of the equivalence relation defined by the question, 
\end{proof}