\subsection{Assignment 3}

\begin{exercise}
    1.
\end{exercise}


\begin{proof}
    We have proved that $(G, \bullet)$ is a group structure in the last assignment. Here, to prove that $(G, \bullet)$ is isomorphic to $(G, *)$. We only need to prove there exists such a bijection $f$ that 

    \begin{equation}\label{eqn:assi3-1-1}
        f(g)\bullet f(h)=f(g*h)\text{.}
    \end{equation}

    It is evident that the inverse operator in $(G, \bullet)$ and $(G, *)$ are the same so we don't need to distinguish $g^{-1}_{*}$ and  $g^{-1}_{\bullet}$. Just let $f$ be the inverse operator, then the equation \ref{eqn:assi3-1-1} satisfies because 

    $$g^{-1} \bullet h^{-1} = h^{-1} * g^{-1} = (g*h)^{-1} $$
\end{proof}

\begin{exercise}
    2.
\end{exercise}

\begin{proof}
    \begin{enumerate}[(i)]
        \item $\mu_\infty$ is a subgroup of $\bbC^{*}$.\\
        Suppose $z_1,z_2\in mu_\infty$, $z_1^{n_1}=1$, $z_2^{n_2}=1$, then $z_1^{n_1n_2}z_2^{n_1n_2}=1$. That means $z_1z_2 \in \mu_\infty$. In addition, $z_1^{-n_1}=1^{-1}=1$, $z_1^{-1} \in \mu_\infty$.
        \item $\mu_\infty=\{e^{2\pi i\frac{k}{n}}| k\in \bbN, n\in \bbN^{+},k<n\}$. On the other hand, $\bbQ/bbZ$ is isomorphic to the rational numbers in $[0,1)$. One can construct such a bijection $f(x)=e^{2\pi ix}$ to map each rational number  in $[0,1)$ which can be expressed as $k/n$, where $k\in \bbN, n\in \bbN^{+},k<n$, to an element in $\mu_\infty$.
    \end{enumerate}
\end{proof}

\begin{exercise}
    3.
\end{exercise}

\begin{proof}
    If $a,b\in [G,G]=C$, then $\bar{a} \cdot \bar{b} = Ca \cdot Cb= Cab=Caba^{-1}b^{-1}ba$. However, $aba^{-1}b^{-1}\in C$, so $Caba^{-1}b^{-1}=C$. Therefore $Cba=\bar{b}\cdot\bar{a}$ and $G/C$ is abelian. And due to $N \triangleleft G$ and $G/N$ abelian, then 
    
    $$\forall a,b\in G, Naba^{-1}b^{-1}=Na\cdot Nb \cdot Na^{-1} \cdot Nb^{-1}$$

    Because $G/N$ abelian, so $Na\cdot Nb \cdot Na^{-1} \cdot Nb^{-1}=Na\cdot Na^{-1} \cdot Nb \cdot Nb^{-1}=K$. Then $Naba^{-1}b^{-1} =N,aba^{-1}b^{-1}\in N$. It indicates that commutator of each two elements $a,b$ is in $N$, while $C$ is generated by commutators, so $C=[G,G]\subset N$.
\end{proof}

\begin{exercise}
    4.
\end{exercise}

\begin{proof}
    \begin{enumerate}[(i)]
        \item Only need to prove that $\sigma$ is an isomorphism, that is, surjective and injective. $\forall A_1, A_2\in \mathrm{SL}_n(\bbC)$ and $A_1\neq A_2$, then $(A_1^{-1})^{t}\neq (A_2^{-1})^{t}$. $\sigma$ is injective. It's similar to prove that its inverse is also injective. so $\sigma$ is surjective.
        \item To prove $\sigma$ is an inner automorphism of $\mathrm{SL}_n (\bbC)$, a matrix $B\in \mathrm{SL}_n (\bbC)$ is needed such that 
        $$\forall A\in \mathrm{SL}_n (\bbC), (A^{-1})^T = BAB^{-1}$$

        When dim is 2 and $A=\begin{pmatrix}
            a & b\\ 
            c & d
            \end{pmatrix}$,  $A^{-1}=\frac{1}{ad-bc}\begin{pmatrix}
                d & -b\\ 
                -c & a
                \end{pmatrix}$, $ad-bc=1$ when $A\in \mathrm{SL}_n (\bbC)$.
                So  $(A^{-1})^T=\begin{pmatrix}
                    d & -c\\ 
                    -b & a
                    \end{pmatrix}$ in this problem. One may choose $B=\begin{pmatrix}
                        0 & -1\\ 
                        1 & 0
                        \end{pmatrix}$ to let the equation holds.
        \item No idea.
    \end{enumerate}
\end{proof}

\begin{exercise}
    5
\end{exercise}

\begin{proof}
    We can compute one by one using the formula:
    \begin{equation}
        \frac{n !}{\prod_{r} r^{n_{r} n_{r} !}}.
        \end{equation}
    $$
\begin{array}{|l|l|l|}\hline 
\text { Partitions for cycle types} & {\text { Representative elements }} & {\text { Size of each conjugacy class }} \\ \hline
1+1+1+1+1+1 & {()} & {1} \\ \hline 
2+1+1+1+1 & {(1,2)} & 15\\ \hline 
2+2+2 & {(1,2)(3,4)(5,6)} & 15\\ \hline 
3+1+1+1 & {(1,2,3)} & {40} \\ \hline 
3+3 & {(1,2,3)(4,5,6)} & {40} \\ \hline 
4+1+1 & {(1,2,3,4)} & {90} \\ \hline 
4+2 & {(1,2,3,4)(5,6)} & {90} \\ \hline 
5+1 & {(1,2,3,4,5)} & {144}\\ \hline 
3+2+1 & {(1,2,3)(4,5)} & {120}  \\ \hline 
6 & {(1,2,3,4,5,6)} & {120}  \\ \hline 
2+2+1+1 & {(1,2)(3,4)} & {45} \\ \hline\end{array}
$$
\end{proof}

\begin{exercise}
    6.
\end{exercise}

\begin{proof}
    $$
    \begin{array}{|l|l|l|}\hline 
        \text { Partitions for cycle types} & {\text { Representative elements }} & {\text { Order }} \\ \hline
        1+1+1+1+1+1 & {()} & 0 \\ \hline 
        2+1+1+1+1 & {(1,2)} & 2\\ \hline 
        2+2+2 & {(1,2)(3,4)(5,6)} & 2\\ \hline 
        3+1+1+1 & {(1,2,3)} & 3 \\ \hline 
        3+3 & {(1,2,3)(4,5,6)} & 3 \\ \hline 
        4+1+1 & {(1,2,3,4)} & 4 \\ \hline 
        4+2 & {(1,2,3,4)(5,6)} & 4 \\ \hline 
        5+1 & {(1,2,3,4,5)} & 5\\ \hline 
        3+2+1 & {(1,2,3)(4,5)} & 6  \\ \hline 
        6 & {(1,2,3,4,5,6)} & 6  \\ \hline 
        2+2+1+1 & {(1,2)(3,4)} & 2 \\ \hline\end{array}
        $$
        The smallest $N$ is 30.
\end{proof}

\begin{exercise}
    7.
\end{exercise}

\begin{proof}
    \begin{enumerate}[(i)]
        \item The matrix can be mapped to an element in $S_3$ in such a manner. That is, the column number $i$ of 1 in each row number $j$ represents part of a permutation $\tau$, $\tau(i)=j$.
        
        Typically, 
        $$
        \left(\begin{array}{lll}{1} & {0} & {0} \\ {0} & {0} & {1} \\ {0} & {1} & {0}\end{array}\right) = (2,3)
        $$ 
        Such a bijection between $S_3$ and $N$ is an isomorphism.
        \item One can consider elements in $N$ as a kind of permutation. If the matrix multiplied on the left side, it can be treated as a row permutation on matrices of $D$. While on the right side, that is a column permutation. The overall effect is equivalent with doing the permutation on the diagonal line sequence.
        Let $$
        A = \left(\begin{array}{lll}
        {a} & {0} & {0} \\ 
        {0} & {b} & {0} \\ 
        {0} & {0} & {c}\end{array}\right) ,
        \tau = \left(\begin{array}{lll}
        {1} & {0} & {0} \\ 
        {0} & {0} & {1} \\ 
        {0} & {1} & {0}\end{array}\right) = (2,3)
        $$ 
        For example, $$\tau A \tau^{-1} = \left(\begin{array}{lll}
            {a} & {0} & {0} \\ 
            {0} & {c} & {0} \\ 
            {0} & {0} & {b}\end{array}\right) $$

        In fact, it is an automorphism on $D$ because it is just a permutation on the sequence of diagonal elements. Therefore, $\tau D \tau^{-1}= D, \forall \tau \in N$
        \item $$\forall A,B\in \mathrm{GL}_3(\bbC), |A^{-1}B^{-1}AB|=|A^{-1}||B^{-1}||A||B|=1,$$ then, $A^{-1}B^{-1}AB\in \mathrm{SL}_3(\bbC)$.
        Inversely, to prove each element $C\in \mathrm{SL}_3(\bbC)$ can be generated by commutators, we need to find the $A,B\in \mathrm{GL}_3(\bbC)$ such that $C=A^{-1}B^{-1}AB$. For simplicity, let $B=I$, then $ACA^{-1}=I$. It indicates that as long as $C$ is diagonalizable, it can be generated by commutators. $C$ is diagonalizable because it is not singular.
    \end{enumerate}
\end{proof}

\begin{exercise}
    8.
\end{exercise}

\begin{proof}
    We know from the Prop. 6-2 in the textbook that $A_n$ is an index two normal subgroup of $S_n$.

    If $H\subset S_n$ is of index 2 then it must be normal and $S_n/H$ is isomorphic to $\bbZ_2^*=\{1,-1\}$. According to the Homomorphism theorem, one can have a surjective homomorphism $f:S_n \rightarrow \bbZ_2^*$ with kernel $Ker\ f = H$. \\
    Note that transpositions in $S_n$ are conjugate, hence $\forall \tau in S_n, f(\tau)\in \bbZ_2^*$ is the same element because $\bbZ_2^*$ is commutative. $S_n$ is generated by transpositions, therefore $\bbZ_2^*$ is generated by $f(\tau)$, while $ord(\bbZ_2^*)=2$. Therefore, $\forall \tau\in S_n, f(\tau)=-1$ and $ker\ f =A_n$. 
\end{proof}

\begin{exercise}
    9.
\end{exercise}

\begin{proof}
    $$ \left(\begin{array}{cc}
        {1} & {0} \\ 
        {0} & {1} \end{array}\right)^1= \left(\begin{array}{ll}
            {1} & {0} \\ 
            {0} & {1} \end{array}\right)$$
    $$ \left(\begin{array}{cc}
        {-1} & {0} \\ 
        {0} & {-1} \end{array}\right)^2= \left(\begin{array}{ll}
            {1} & {0} \\ 
            {0} & {1} \end{array}\right)$$
    $$ \left(\begin{array}{cc}
        {0} & {-1} \\ 
        {1} & {-1} \end{array}\right)^3= \left(\begin{array}{ll}
            {1} & {0} \\ 
            {0} & {1} \end{array}\right)$$
    $$ \left(\begin{array}{cc}
        {0} & {-1} \\ 
        {-1} & {0} \end{array}\right)^4= \left(\begin{array}{ll}
            {1} & {0} \\ 
            {0} & {1} \end{array}\right)$$
    
    $$ \left(\begin{array}{cc}
        {0} & {-1} \\ 
        {1} & {1} \end{array}\right)^6= \left(\begin{array}{ll}
            {1} & {0} \\ 
            {0} & {1} \end{array}\right)$$
    $$ \left(\begin{array}{cc}
        {2} & {1} \\ 
        {1} & {1} \end{array}\right)^\infty\neq \left(\begin{array}{ll}
            {1} & {0} \\ 
            {0} & {1} \end{array}\right)$$
\end{proof}