\subsection{Assignment 8}

\begin{exercise}
    1.
\end{exercise}

\begin{proof}

    It is clear that $(x^2+1)$ is an ideal, while the quotient ring's unity is $1+(x^2+1)$. To prove that the quotient ring is field first, we need to show that $$\forall f(x)\in \bbR[x], \exists g(x)\in \bbR[x].$$
    It turns out to be proving 
    $$f(x)g(x)-1\in (x^2+1). $$


    $1+x^2$ is irreducible (in $\bbR[x]$) and if $x^2+1\nmid f(x)$ then\\
    $\Rightarrow$ $x^2+1$ and $f(x)$ are coprime in $\bbR[x]$. \\
    $\Rightarrow$ Therefore thre are $g(x),h(x)\in \bbR[x]$ such that $$f(x)g(x)+(x^2+1)h(x)=1.$$ So, $f(x)g(x)\in 1+(x^2+1)$. And the quotient ring is really a field.

    To investigate the quotient ring structure, consider a ring substitution homomorphism $\phi:\bbR[x]\rightarrow\bbC$ with $\phi(f)=f(i)$. This mapping is surjective, since $\bbC=\bbR[i]=\bbR(i)$ (ring adjoint equals field adjoint here). Thus by the Homomorphism theorem, $\bbR[x]/\mathrm{ker} \phi$ is isomorphic to $\bbC$. The kernel is a principal ideal generated by the minimal polynomial $x^2+1$ of $i$ over $\bbR$. 
\end{proof}

\begin{exercise}
    2.
\end{exercise}

\begin{proof}

    \begin{enumerate}[(1)]
        \item     $I_1$, $I_2$ and $I_3$ are coprime,\\
        $\Rightarrow$ $I_1+I_2=R$, so 
        $$\exists a_1\in I_1, b_1 \in I_2, \textit{s.t.} 1=a_1+b_1, $$
        which is the same for $I_1,I_3$ with similar $a_2\in I_1,b_2\in I_3$, \textit{s.t.} $1=a_2+b_2$.\\
        $$\Rightarrow 1=(a_1+b_1)(a_2+b_2)=a_1a_2+a_1b_2+a_2b_1+a_2b_2,$$
        in which $a_1a_2+a_1b_2+a_2b_1\in I_1, b_1b_2\in I_2I_3$
        $$\Rightarrow I_1+I_2I_3=R.$$
        By induction, we can go further to prove $I_1$ and $I_2I_3I_4\dots$ are coprime.
        \item We still go with the simple case firstly, proving $I_1I_2=I_1\cap I_2$. $I_1I_2\subset I_1\cap I_2$ is evident.\\
        If $x\in I_1\cap I_2$, suppose $1=a+b,a\in I_1,b\in I_2$, then 
        $$x=x\cdot 1=x(a+b)=xa+xb,$$ in which $xa\in I_1I_2$, $xb\in I_1I_2.$ $\Rightarrow x\in I_1I_2$, $I_1\cap I_2\subset I_1I_2.$\\
        By induction, 
        $$I_1I_2\dots I_k=I_1\cap I_2\cap \dots \cap I_k$$
    \end{enumerate}

\end{proof}

\begin{exercise}
3.
\end{exercise}

\begin{proof}
    \begin{enumerate}[(1)]
        \item First of all, we prove the theorem: 
        \begin{theorem}[Chinese remainder theorem]
            $R$ is a commutative ring with unity, $I_1,I_2,\dots I_n$ are pairwise coprime ideals of $R$ and $a_1,a_2,\dots a_n\in R$. Then, $$\exists a\in R\quad \textit{s.t.} a\equiv a_i \pmod{I_i}\quad \forall i$$
        \end{theorem}
        \begin{proof}
            We only need to find the elements, $y_1,y_2,\dots,y_n$, 
            $$ \textit{s.t.} \forall i, y_i \equiv 1 \pmod{I_1}, y_i\equiv 0 \pmod{I_j} (i\neq j). $$
            As long as we have these elements, the answer is easy to be constructed by $$a=a_1y_1+\dots a_ny_n.$$\\
            So how to find these elements?\\
            For the element $y_1$, because $I_1+I_j=R$, $\exists b_j\in I_1,c_j\in I_j$, \textit{s.t.} $b_j+c_j=1$. Let's gather all the $c_j$, and let $y_1=c_2\dots c_n=(1-b_2)\cdots (1-b_n)\in I_2\dots I_n$. Hence, $y_1\in I_2 \cap \cdots \cap I_n \cdot y_1\equiv 1 \pmod{I_1} \text{and} \equiv 0 \pmod{I_j} (j\neq i)$. The procedure to calculate other $y_i$ is similar.
        \end{proof}
        \item Next to prove $$R/I\cong R/I_1\times R/I_2  \dots \times R/I_n.$$
        Let $f:R\rightarrow R/I_1\times \cdots \times R/I_n$, $f(r)=(r+I_1, \cdots, r+I_n)$. And we know this is a surjective ring homomorphism from the Chinese remainder theorem. $\textrm{ker } f=\{r\in R | r\in I_1 \cap \cdots I_n\}.$ By utility of the last exercise, we know that 
        $$R/(I_1\cap I_2 \cdots \cap I_k)\cong R/I_1 \times R/I_2 \cdots R/I_k.$$
    \end{enumerate}
\end{proof}

\begin{exercise}
    4.
\end{exercise}

\begin{proof}
    From $f$, we can induce a ring homomorphism $g$ from $R$ to $S/Q$ by postcomposing with the natural projection map $R \rightarrow S/Q$.
    $$g: R \overset{f}{\rightarrow} S\rightarrow S/Q $$
    Now $a\in \textrm{ker } g$ if and
only if $f(a) \in P$, so using the first isomorphism theorem we see that $g$ induces an
isomorphism from $R/f^{-1}(Q)$ to a subring of $S/Q$. Since the latter is an integral domain, $R/f^{-1}(Q)$ must be an integral domain too.
\end{proof}

\begin{exercise}
    5.
\end{exercise}

\begin{proof}
    \begin{enumerate}
        \item Not yet proved, $$\begin{cases}
            \bbC [x]/(x^2), prime\\
            (x)/(x^2), maximal\\
            \overline{0}, prime
        \end{cases}$$
        \item Not yet proved.
    \end{enumerate}
\end{proof}

\begin{exercise}
    6.
\end{exercise}

\begin{proof}
    \begin{enumerate}[(i)]
        \item $$\begin{cases}
            \overline{0}, \\
            (x+1)/(x^2-1), (x-1)/(x^2-1) \text{maximal}\\
            (x+1)/(x^2-1), (x-1)/(x^2-1), R \text{prime}
        \end{cases}$$
        \item $$R\cong \bbC \oplus \bbC \Rightarrow R^{\times} \cong (\bbC \oplus \bbC)^* \cong \bbC^* \oplus \bbC^*$$
    \end{enumerate}

\end{proof}