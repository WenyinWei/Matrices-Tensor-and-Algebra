\subsection{Nonlinear Optics Appendix}

第三章,折射率随光强改变的效应

\begin{exercise}

1. 

$n_0=1.6,\lambda = 1\mu m$,$ \chi_{sl}^{(3)}=2.66*10^{-20} (m/V)^2$

求非线性折射率 $n_2$, 若光强为 $1MW/cm^2$,求折射率的增量 $\Delta n$.

\end{exercise}

\begin{proof}
代入非线性折射率公式,
$$n_2=\frac{3\mu_0 c}{4n_{10}^{2}} \chi_{sl}^{(3)}=2.94*10^{-18} W/m^2 $$
$$\Delta n = n_2 I = 2.94*10^{-8} $$
\end{proof}

\begin{exercise}
2.
$$\tau \frac{d\Delta n}{dt}+\Delta n=n_2 I(t)$$    
\end{exercise}

\begin{proof}
首先解出上述方程的齐次方程,解为 $e^{-t/\tau}$.
然后令系数变为函数$$\tau \frac{d C(t)e^{-t/\tau}}{dt}+ C(t)e^{-t/\tau}=n_2I(t)$$
解得前面的系数随时间变化$$C^\prime=e^{t/\tau}n_2I(t)/\tau $$
\end{proof}


\begin{exercise}
3.
\end{exercise}

\begin{proof}
$k^{NL}=\frac{3\omega_1}{8cn_1} \chi^{(3)} |A_1|^2$
$$\Delta k_{3}= k_3-k_3^{P}=k_3-3k_1^P=k_3-3k_1-3k^{NL}$$

相干长度 $L_c=\pi / \Delta k_{3}$
\end{proof}

\begin{exercise}
4.
\end{exercise}

\begin{proof}
从公式中可见,当 $\theta = \pi/4, 3\pi/4, -\pi/4, -3\pi/4$ 时取最大值。
\end{proof}

\begin{exercise}
5. 
\end{exercise}

\begin{proof}

\end{proof}

\begin{exercise}
    6. 
\end{exercise}

\begin{proof}
    $$
    R=\frac{\left|A_{4}(0)\right|^{2}}{\left|A_{s}(0)\right|^{2}}=\tan ^{2}(|K| L)
    $$
    小 R 近似下,$
    R\approx(|K| L)^2
    $.

    而$$
K=\frac{3 \omega}{4 c n_{0}} \chi^{(3)} A_{1}(0) A_{2}(0)
, 
I=\frac{\mathrm{c} n \varepsilon_{0}}{2}|A|^{2}
$$
将 K 式中的振幅振幅替换为光强,可得
$$
\chi^{(3)}=\frac{2 \varepsilon_{0} c^{2} n_{0}^{2} \sqrt{R}}{3 \omega L \sqrt{I_{1} I_{2}}}
$$
\end{proof}
    
\begin{exercise}
    7. 
\end{exercise}

\begin{proof}

\end{proof}

\begin{exercise}
8. 
\end{exercise}

\begin{proof}

\end{proof}