\subsection{Nonlinear Optics Assignment 4}

第三章,折射率随光强改变的效应

\begin{exercise}

1. 

$n_0=1.6,\lambda = 1\mu m$, $ \chi_{sl}^{(3)}=2.66*10^{-20} (m/V)^2$

求非线性折射率 $n_2$, 若光强为 $1MW/cm^2$,求折射率的增量 $\Delta n$.

\end{exercise}

\begin{proof}
代入非线性折射率公式,
$$n_2=\frac{3\mu_0 c}{4n_{10}^{2}} \chi_{sl}^{(3)}=2.94*10^{-18} W/m^2 $$
$$\Delta n = n_2 I = 2.94*10^{-8} $$

用泵浦-检测光系统,通过光学克尔效应检测。
\end{proof}

\begin{exercise}
2.
$$\tau \frac{d\Delta n}{dt}+\Delta n=n_2 I(t)$$    
\end{exercise}

\begin{proof}
首先解出上述方程的齐次方程,解为 $C e^{-t/\tau}$.
然后通过变系数法 $$\tau \frac{d C(t)e^{-t/\tau}}{dt}+ C(t)e^{-t/\tau}=n_2I(t)$$
解得前面的系数随时间变化$$C^\prime=e^{t/\tau}n_2I(t)/\tau $$


只要 $\Delta n$ 的变化趋势可以写为 
$$(C(0)+\int_0^t e^{t/\tau}n_2I(t)/\tau)e^{-t/\tau}$$

\end{proof}


\begin{exercise}
3.
\end{exercise}

\begin{proof}
基频光和三倍频都有各自的受光强调制的折射率,
$$
\begin{aligned}
&n=n_{0}+n_{2}(\omega) I\\
&n^{\prime}=n_{0}^{\prime}+n_{2}^{\prime} (3\omega,\omega) I
\end{aligned}
$$
$$
\begin{aligned}
&n_{2}\left(\omega\right)=\frac{3 \mu_{0} c}{4 n_{0}^{2}} \chi^{(3)}\left(-\omega_{1} ; \omega_{1},-\omega_{1}, \omega_{1}\right)\\
&n_{2}^\prime\left(3\omega, \omega\right)=\frac{3 \mu_{0} c}{2 n_{0} n_{0}^\prime} \chi^{(3)}\left(-3\omega ; \omega,-\omega, 3\omega\right)
\end{aligned}
$$


相位失匹配量:
$$
\begin{aligned}
\Delta k+\Delta k^{NL} &=k_{3}-3 k_{1} \\
&=\frac{n^{\prime}(3 \omega)}{c}-3 \frac{n \cdot \omega}{c} \\
&=\frac{3 \omega}{c}\left(n_{0}^{\prime}+n_{2}^{\prime} I-n_{0}-n_{2} I\right)
\end{aligned}
$$
其中 $\Delta k^{NL} = 3 \omega c^{-1} \left(n_{2}^{\prime} I-n_{2} I\right)$

相干长度 $L_c=\pi / \Delta k^{NL}$
\end{proof}

\begin{exercise}
4.
\end{exercise}

\begin{proof}
从公式中可见,当 $\theta = \pi/4, 3\pi/4, -\pi/4, -3\pi/4$ 时取最大值。

我们在普通光学中学过,如果一束线偏振光经过正交的起偏器和检偏器,中间还有一个波晶片,
波晶片光轴与检偏器呈 $\alpha$ 角,
波晶片在 o, e 两光线偏振方向导致 $\Delta \Phi$ 的相位差的话,那么出射光振幅应该是

$$
\begin{aligned}
A^2&=A_{1o}^2+A_{1e}^2+2A_{1o}A_{1e}\cos(\Delta \Phi +\pi)\\
&=(A_1 \sin \alpha\cos\alpha)^2+(A_1 \cos\alpha\sin\alpha)^2+
2(A_1\sin\alpha\cos\alpha)(A_1 \cos\alpha\sin\alpha)\cos(\Delta \Phi +\pi)\\
&=2A_1^2\sin^2\alpha\cos^2\alpha(1+\cos{\Delta\Phi+\pi})\\
&=A_1^2\sin2\alpha\cos^2(\Delta\Phi/2+\pi/2)\\
&=A_1^2\sin2\alpha\sin^2(\Delta\Phi/2)
\end{aligned}
$$

那么在本题中,做一个类比,相当于入射弱检测光和泵浦光 y 偏振方向成 $\alpha$ 角,从而易得出光强公式.

$$
I_{D}\left(\omega_{2}\right)=I\left(\omega_{2}\right) \sin ^{2} 2 \theta \sin ^{2}\left[\pi B\left(\lambda_{2}\right) I_{P}\left(\omega_{1}\right) L\right]
$$

\end{proof}

\begin{exercise}
5. 
\end{exercise}

\begin{proof}
由 Maker-Terhune 表示可以知道下面表达式对各向同性介质成立:

$$
\begin{aligned}
&\Tens{\chi}^{(3)} \vdots \Vect{A_{1}} \Vect{A_{2}} \Vect{A_{3}}\\
=&\chi_{1221}^{(3)}(\Vect{A_{1}} \cdot \Vect{A_{2}}) \Vect{A_{3}}\\
+&\chi_{1122}^{(3)}(\Vect{A_{2}} \cdot\Vect{A_{3}}) \Vect{A_{1}}\\
+&\chi_{1212}^{(3)}(\Vect{A_{3}}\cdot\Vect{A_{1}}) \Vect{A_{2}}
\end{aligned}
$$




在书中我们原本为了简单起见,假设线偏振光全都朝一个方向,在这里我们对原式 (见下 (书中 8.a)) 进行扩展.
$$
P_{4, a}^{(3)}(\vec{r}, \omega)=\frac{3 \varepsilon_{0}}{2} \chi_{1111}^{(3)}(-\omega ; \omega, \omega,-\omega) A_{1} A_{2} A_{s}^{*} e^{i\left(\vec{k}_{1}+\vec{k}_{2}-\vec{k}_{3}\right) \cdot \vec{r}}
$$

$$
\Vect{P}_{4}^{(3)}(\Vect{r}, \omega)
=\frac{3 \varepsilon_{0}}{2} \Tens{\chi}^{(3)}(-\omega ; \omega, \omega,-\omega)
\vdots \Vect{A}_{1} \Vect{A}_{2} \Vect{A}_{s}^{*} e^{-i\Vect{k}_{3}\cdot \Vect{r}}
$$
其中我们挑出来张量点积展开,
$$
\begin{aligned}
&\Tens{\chi}^{(3)} \vdots \Vect{A_{1}} \Vect{A_{2}} \Vect{A_{s}^{*}}\\
=&\chi_{1221}^{(3)}(\Vect{A_{1}} \cdot \Vect{A_{2}}) \Vect{A_{s}^{*}}\\
+&\chi_{1122}^{(3)}(\Vect{A_{2}} \cdot\Vect{A_{s}^{*}}) \Vect{A_{1}}\\
+&\chi_{1212}^{(3)}(\Vect{A_{s}^{*}}\cdot\Vect{A_{1}}) \Vect{A_{2}}
\end{aligned}
$$
对该四阶张量 $\Tens{\chi}^{(3)}(-\omega ; \omega, \omega,-\omega)$, 我们有对称性 $\chi_{1122}^{(3)}=\chi_{1212}^{(3)}$, 于是可得最终结果:

$$
\begin{array}{c}
{\Vect{P}_{4}^{(3)}=\left\{A\left[\left(\Vect{A}_{1} \cdot \Vect{A}_{s}^{*}\right) \Vect{A}_{2}+\left(\Vect{A}_{2} \cdot \Vect{A}_{s}^{*}\right) \Vect{A}_{1}\right]+C\left(\Vect{A}_{1} \cdot \Vect{A}_{2}\right) \Vect{A}_{s}^{*}\right\} e^{-i \Vect{k}_{s} \cdot\Vect{z}}} \\
{A=\frac{3 \varepsilon_{0}}{2} \chi_{1122}^{(3)}(-\omega ; \omega, \omega,-\omega), \quad C=\frac{3 \varepsilon_{0}}{2} \chi_{1221}^{(3)}(-\omega ; \omega, \omega,-\omega)}
\end{array}
$$
\end{proof}

\begin{exercise}
    6. 
\end{exercise}

\begin{proof}
    $$ 
K=\frac{3 \omega}{4 c n_{0}} \chi^{(3)} A_{1}(0) A_{2}(0)
, 
I=\frac{\mathrm{c} n \varepsilon_{0}}{2}|A|^{2}
$$
将 K 式中的振幅振幅替换为光强,可得
$$
\chi^{(3)}=\frac{2 \varepsilon_{0} c^{2} n_{0}^{2} \sqrt{R}}{3 \omega L \sqrt{I_{1} I_{2}}}
$$
\end{proof}
    
\begin{exercise}
    7. 
\end{exercise}

\begin{proof}
基本参数:
$$
\begin{aligned}
&\lambda= \SI{1.0e-6}{m}\\
&n_{0}=1.49\\
&n_{2}=\SI{0.7 e-18 }{m^2/W}\\
&w_{0}=\SI{500e-6}{m}\\
&k=\omega[\lambda] n_{0} / c\\
&P / P_{c r}=1.5
\end{aligned}
$$
无象差的临界光功率及焦距
$$
P_{c r}=\frac{\lambda_{0}^{2}}{8 \pi n_{0} n_{2}}=\SI{3.81484e4}{W/m^2}
$$
$$
z_{F}=\frac{k w_{0}^{2}}{\sqrt{\frac{P}{P_{c r}}-1}}=\SI{3.30995}{m}
$$
有象差的临界光功率及焦距
$$
\begin{aligned}
&P_{c r}=\frac{\pi\left(1.22 \lambda_{0}\right)^{2}}{2 n_{0} n_{2}}=\SI{2.24159e6}{W/m^2}\\
&z_{F}=\frac{0.43 k w_{0}^{2}}{\left\{\left[\left(P / P_{c r}\right)^{1 / 2}-0.852\right]^{2}-0.0219\right\}^{1 / 2}}=2.94178
\end{aligned}
$$

\end{proof}

\begin{exercise}
8. 
\end{exercise}

\begin{proof}
$$
\omega(t)=\omega_{0}+\frac{2 \gamma A_{0}^{2} z_{0} t}{\tau^{2}} \exp \left(-\frac{t^{2}}{\tau^{2}}\right)
$$
对瞬态频率求极值,
$$
\begin{array}{l}
{\frac{2 \gamma A_{0}^2 z_{0}}{-\tau^{2}}e^{t^{2}/\tau^2}+\frac{2 \gamma A_{0}^{2} z_{0} t}{\tau^{2}} e^{-t^{2} / \tau^{2}}\left(-\frac{2 t}{\tau^{2}}\right)} \\
{=\frac{2 \gamma A_{0}^{2} z_{0}}{\tau^{2}} e^{-t^{2} / \tau^{2}}\left(1+\frac{-2 t^{2}}{\tau^{2}}\right)=\frac{2 \gamma A_{0}^{2} z_{0}}{\tau^{2}}\left(\frac{\tau^{2} -2t^{2}}{\tau^{2}}\right)}
\end{array}
$$
由上式可知 $t=\pm \tau/\sqrt{2}$ 时取极值,
$$
\begin{aligned}
& \Delta \omega=2\left(\frac{2 \gamma A_0^{2} z_{0} \frac{\tau}{\sqrt{2}}}{\tau^{2}} e^{-1 / 2}\right)=2 \sqrt{2} \frac{\gamma A_{0}^{2} z_{0}}{\tau \sqrt{e}}=2 \sqrt{\frac{2}{e}} \frac{\omega_{0} n_{2}^{\prime} A_{0}^{2} z_{0}}{c \tau} \\
=& 2 \sqrt{\frac{2}{2}} \frac{2 \pi c^{\prime} n_{2}^{\prime} A_{0}^{2} z_{0}}{\lambda c \tau}=4 \pi \sqrt{\frac{2}{e}} \frac{n_{2}^{\prime} A_{0}^{2} z_{0}}{\lambda \tau}=3.431 \pi \frac{n_{2}^{\prime} A_{0}^{2} z}{\lambda \tau}
\end{aligned}
$$

如果相干峰间距为 $\omega_\Delta$
$$
\begin{aligned}
&w_{\Delta}=2 \pi f=\frac{2 \pi}{\tau}\\
&N_S+N_A=\frac{\Delta w}{w_{\Delta}} = 1.7 n_{2}^{\prime} A_{0}^{2} z / \lambda 
\end{aligned}
$$
功率谱中红移部分占多数,故$N_S\approx n_{2}^{\prime} A_{0}^{2} z / \lambda $
\end{proof}