\subsection{Assignment 2}

\begin{exercise}
1. Let $G$ be a group. Prove that the order of $g$ is equal to the order of $g^{-1}$ for all $g\in G$.
\end{exercise}

\begin{proof}
We extract a countable set $\{x|x=1/2^n,n\in \bbN^+ \}=\{1/2,1/2^2,1/2^3,\cdots\}$ first from $(0,1)$. Map $1/2$ to $1$, that is, $1/2^0$. Map $1/2^2$ to $1/2$, while $1/2^n$ is mapped to $1/2^{n-1}$. This is the map we need. While for other domain in $(0,1)$, just map them to themselves. 
\end{proof}

\begin{exercise}
    Let $G$ be a group with group law $\bullet$, that is, the product of $g$ and h is $g \bullet h$ for all $g, h \in G$.
    Define a new binary operation $* : G \times G \rightarrow $G as $g * h := h \bullet g$. Prove that this defines a group structure on G (called the \textit{opposite group}).
\end{exercise}

\begin{exercise}
    Let $G$ be a group with subgroups $K$ and $H$ such that $K \leq H \leq G$. Prove the following statements.
    \begin{itemize}
        \item  The index $[G : K]$ is finite if and only if $[G : H]$ and $[H : K]$ are finite.
        \item If $[G : K]$ is finite, then $[G : K] = [G : H] \times [H : K]$.

    \end{itemize}
\end{exercise}

\begin{exercise}
    Let $G$ be a group with subgroups $K$ and $H$. Prove the following statements
    \begin{itemize}
        \item  If $K$ is normal in $G$, then $H \bigcap K$ is normal in $H$.
        \item Give an example of $K \leq H \leq G$ such that $K$ is normal in $H$ and $H$ is normal in $G$ but $K$ is not normal in $G$.
        
    \end{itemize}
\end{exercise}

\begin{exercise}
    Let $G$ be a finite subgroup of $\mathrm{GL}_n(\bbC)$ (the group of $n$ by $n$ invertible complex matrices).
Prove the following statements.
\begin{itemize}
    \item Every $g \in G$ is diagonalizable.
    \item  If $G$ is abelian, then all elements in $G$ can be simultaneously diagonalized.
\end{itemize}
\end{exercise}

\begin{exercise}
    Let $G$ be a finite group of order $pq$, where $p$ and $q$ are distinct prime numbers. Prove the
following statements.
\begin{itemize}
    \item The center $C(G)$ is either $\{e\}$ or $G$.
    \item  There exist elements $g$ and $h$ of $G$ such that $ord(g) = p$ and $ord(h) = q$ (hint:
    consider the equivalence relation on $G: g_1 \sim g_2$ if and only if $h^{-1} g_{1} h=g_{2}$ for some
    $h \in G$).
\end{itemize}
\end{exercise}