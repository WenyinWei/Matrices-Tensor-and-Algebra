\documentclass[10pt]{article}
% \pdfoutput=1
\usepackage{NotesTeX,lipsum}
%\usepackage{showframe}

\usepackage{xcolor}
\renewcommand{\emph}[1]{\textcolor{purple}{\textit{#1}}}
% Too large line blank between neighboring lines.
\setenumerate[1]{itemsep=0pt,partopsep=0pt,parsep=\parskip,topsep=5pt}
\setitemize[1]{itemsep=0pt,partopsep=0pt,parsep=\parskip,topsep=5pt}
\setdescription{itemsep=0pt,partopsep=0pt,parsep=\parskip,topsep=5pt}
\usepackage{xeCJK}
\setCJKmainfont[BoldFont=SimHei]{SimSun}
\setCJKfamilyfont{hei}{SimHei}
\setCJKfamilyfont{kai}{KaiTi}
\setCJKfamilyfont{fang}{FangSong}
\newcommand{\hei}{\CJKfamily{hei}}
\newcommand{\kai}{\CJKfamily{kai}}
\newcommand{\fang}{\CJKfamily{fang}}

\title{\begin{center}{\Huge \textit{Matrices, Tensor and Algebra}}\end{center}}
\author{Wenyin Wei\footnote{\href{https://wenyin.xyz/}{\textit{My Personal Website}}}}


\affiliation{
Tsinghua University\\
Department of Engineering Physics
}

\emailAdd{weiwy16@mails.tsinghua.edu.cn}

\begin{document}
	\maketitle
	\flushbottom
	\newpage
	\pagestyle{fancynotes}

	% \part{Algebra}
	% The part mainly comes from the notes of Prof. 许俊彦 2019 清华 抽象代数。
	% \section{Basics of Set Theory}
	% 

\subsection{Constructions of sets}

\begin{enumerate}
    \item Given a set $S$. The power set $P(S)$ of $S$ is defined to be the set of all subsets of $S$. \textit{E.g.},
    $S = \{1, 2, 3\}$, $P(S) = \{\emptyset, \{1\}, \{2\}, \{3\}, \{1, 2\}, \{1, 3\}, \{2, 3\}, \{1, 2, 3\}\}$.
    \item  Let I be a non-empty set and for each $i \in I$, one has a set $A_i$ Then one can define
    \begin{itemize}
        \item $\bigcap_{i \in I} A_{i}$, the \textcolor{purple}{\textit{intersection}} of all sets $A_{i}$.
        \item $\bigcup_{i \in I} A_{i}$, the \textcolor{purple}{\textit{union}} of all sets $A_{i}$.
        \item $\prod_{i \in I} A_{i}$, the \textcolor{purple}{\textit{Cartesian product}} of all sets $A_{i}$.
    \end{itemize}
\end{enumerate}

\begin{definition}
    Let $A$ and $B$ be two non-empty sets. A function $f : A \rightarrow B$ is a rule that assigns to every
    element $a$ of $A$ a unique element $b$ of $B$. Equivalently, the function $f$ can be described
    as a subset $\Gamma \subset A \times B$ satisfying the following conditions:
    \begin{enumerate}
        \item $\forall a \in A$, $\in b \in B$ such that $(a, b) \in \Gamma$;
        \item if $(a, b)$ and $(a, b^{\prime})$ belong to $\Gamma$, then $b = b$
    \end{enumerate}
\end{definition}


Given a function $f : A \rightarrow B$, one can talk about whether $f$ is injective, surjective,
bijective, inverse function of $f$ (if bijective), images and preimages.

Given two functions $f : A \rightarrow B$ and $g : B \rightarrow C$, one can define the composition function
$g \circ f : A \rightarrow C$ that maps a to $g(f(a))$.

\begin{lemma}
    Any set $S$ is not bijective to its power set $P(S)$.
\end{lemma}

\subsection{Equivalence relations}
\begin{definition}
    Let $S$ be a non-empty set. A \textcolor{purple}{\textit{relation $\sim$ on $S$}} is defined by a subset $R \subset  S \times  S$ in
the following sense: we say $a \sim b$ iff $(a, b) \in R$. A set $S$ equipped with a relation $\sim$ is
denoted by $(S, \sim)$.
\end{definition}

\begin{definition}
Given $(S, \sim)$, the relation $\sim$ is said to be an \textcolor{purple}{\textit{equivalence relation}} if the following conditions
hold:
\begin{enumerate}
    \item $\forall x \in S$, $x \sim x$ (reflexive);
    \item if $x \sim y$, then $y \sim x$ (symmetric);
    \item if $x \sim y$ and $y \sim z$, then $x \sim z$ (transitive).
\end{enumerate}

\end{definition}




If $S$ is equipped with an equivalence relation $\sim$ and $x, y \in S$, we say x is equivalent to
$y$ if $x \sim y$. The subset $[x] := \{y \in S| y \sim x\}$ of $S$ consisting of all elements equivalent
to $x$ is called the equivalence class of $x$. The set of all equivalence classes is denoted by
$S/ \sim$.

Equivalence relations occur everywhere, \textit{e.g.}, one can define an equivalence relation on
$Z$ by saying that $m \sim n$ if $m - n$ is divisible by $2$, in this case there are two equivalence
classes: the class of even numbers and the class of odd numbers. Another example, let
$S$ be the set of Chinese people and we define two Chinese $x$ and $y$ to be equivalent iff
$x$ and $y$ have the same surname, in this case the equivalence classes correspond to the
surnames of Chinese.

\begin{lemma}
A fundamental FACT about equivalence relation $(S, \sim): \forall x, y \in S$, either $[x] = [y]$ or
$[x] \bigcap [y] = \emptyset$. This fact implies that $S$ is the disjoint union of (or partitioned by) the
equivalence classes of S. Equivalently, one can defines an equivalence relation $\sim$ on $S$ by
partitioning $S$ into disjoint non-empty subsets and saying that $ x \sim y$ iff $x$ and $y$ belong
to the same partition.
\end{lemma}

Given $S$ with equivalence relation $\sim$, one has a natural (surjective) function $\pi: S \rightarrow S/ \sim$
mapping $x \in S$ to $[x] \in S/ \sim$.

\subsection{Partial Order Set and \textit{Zorn's Lemma}}

\begin{definition}
    Let $S$ be a set equipped with a relation $\leq$. We say that $(S, \leq)$ is a \textcolor{purple}{\textit{partial order set}} if
    the following conditions hold:
    \begin{enumerate}
        \item $\forall x \in S$, $x \leq x$;
        \item if $x \leq  y$ and $y \leq  x$, then $x = y$;
        \item if $x \leq  y$ and $y \leq  z$, then $x \leq z$.
    \end{enumerate}

\end{definition}

\begin{definition}
A partial order set $(S, \leq )$ is said to be a \textcolor{purple}{\textit{total order set}} if $\forall x, y \in S$, either $x \leq  y$ or
    $y \leq  x$ holds.
\end{definition}

\begin{example}
    Given a set $S$, then $(P(S), \subset )$ is a partial order set (but not a total order set in general).
\end{example}

\begin{definition}
    \textbf{Chain, Upper Bound and Maximal Element}\\
    \begin{enumerate}
        \item Given a partial order set $(S, \leq )$, a subset $C \subset  S$ is called a \textcolor{purple}{\textit{chain}} if $\forall x, y \in C$ either
        $x \leq  y $or $y \leq  x$ holds.
        
        \item Given a partial order set $(S, \leq )$ and a subset $T \subset  S$, an element $x \in S$ is called an \textcolor{purple}{\textit{upper
        bound}} of $T$ if$ \forall y \in T$, $y \leq  x$ holds.
        
        \item Given a partial order set $(S, \leq )$, an element $x \in S$ is called a \textcolor{purple}{\textit{maximal element}} if there
        does not exist $y \in S $ such that $x \leq  y$ and $y = x$. Remark: $(S, \leq )$ can have multiple
        maximal elements.
    \end{enumerate}
\end{definition}


\begin{lemma}
\textcolor{purple}{\textbf{Zorn’s lemma}}\\
 Let $(S, \leq )$ be a partial order set. If every chain $C \subset  S$ has an upper
bound in $S$, then $S$ has a maximal element.

One can prove by \textit{Zorn’s lemma} that every non-zero vector space has a basis.
\end{lemma}
	% 

\subsection{Group Theory 1}
\subsubsection{Groups and subgroups}.
\begin{definition}
    A \emph{group} is a non-empty set G together with a binary operation $G \times G \rightarrow G$ (called
“multiplication” or “group law”) that sends $(g_1, g_2)$ to $ g_1 \cdot g_2$ (or omitting the $\cdot$ and just
write $g_1g_2$ for simplicity) such that the following axioms are satisfied.

(i) (Associativity): $g_1(g_2g_3) = (g_1 g_2)g_3$ for $\forall g_1, g_2, g_3 \in G$.

(ii) (Existence of identity): $\exists e \in G$ such that $eg = g = ge, \forall g \in G$.

(iii) (Existence of inverse): $\forall g \in G$, $\exists h \in G$ such that $gh = e = hg$.
\end{definition}

\begin{remark}
    A few remarks about this notion.

    (a) Associativity allows us to “remove the brackets”: $g_1(g_2g_3)$ can simply be represented by $g_1g_2g_3$.

(b) If both $e$ and $e^{\prime}$
satisfy axiom (ii), then $e = ee^{\prime} = e^{\prime}$. Thus $e$ is unique and is called
the identity element.

(c) The element $h$ in axiom (iii) is uniquely determined by $g$: if both $h$ and $h^{\prime}$ satisfy (iii) for $g$, then we obtain $gh = e \Rightarrow h^{\prime}gh=h^{\prime}\Rightarrow eh=h^{\prime}\Rightarrow h=h^{\prime}$. Hence, it makes
sense to say that $h$ is the inverse element of $g$ and write it as $g^{\prime}$

(d) Suppose we fix two elements $g,h \in G$ and let $x$ be a "variable" in $G$. THen the equation $gx=g$ (resp. $xg=h$) has a unique solution $x = g^{-1}h$ (resp. $hg^{-1}$).

(e) Fix $g \in G$ and define a map (left multiplication by $g$) $L_g : G \Rightarrow G$ such that $h \mapsto gh$.
By (d), this map is a bijection from G to itself. Similarly, the right multiplication
by $g$ map $R_g : G \rightarrow G$ is also bijective.
\end{remark}

\begin{definition}
    A subset $H \subset G$ is called a \emph{subgroup} of $G$, denoted $H \leq G$, if the following conditions hold. 
    \begin{enumerate}
        \item (Closed under multiplication): $h_1h_2 \in H$, $\forall h_1,h_2 \in H$.
        \item (Existence of identity): $e \in H$.
        \item (Existence of inverse): $\forall h \in H$, the inverse $h^{-1}\in H$.
    \end{enumerate}
\end{definition}

\begin{example}
    Examples of groups and subgroups:
    \begin{enumerate}
        \item Group with one element: $\{e\}$.
        \item Under usual addition law: $\{0\} \leq 4\bbZ \leq 2\bbZ \leq \bbZ \leq \bbQ \leq \bbR \leq \bbC$.
        \item Residue classes $a$ (mod $n$) under usual addition law: $\bbZ / n\bbZ$.
        \item Residue classes $a$ (mod $n$) such that $(a, n) = 1$ under usual multiplication law:
        $(\bbZ/n\bbZ)^{*}$.
        \item Fix $n \in \bbN$, invertible matrices under matrix multiplication: $\mathrm{GL}_n(\bbQ) \leq \mathrm{GL}_n(\bbR) \leq \mathrm{GL}_n(\bbC) $. When $n=1$, we obtain $\bbQ^* \leq \bbR^* \leq \bbC^*$ under usual number multiplication.
        \item Let $\Sigma$ be a non-empty set and $Perm(\Sigma)$ be the set of bijective functions from $\Sigma$ to itself. Then $Perm(\Sigma)$ is group with group law given by composition of functions:
        $$
        f \cdot g :=f \circ g : \Sigma \stackrel{g}{\rightarrow} \Sigma \stackrel{f}{\rightarrow} \Sigma
        $$
        If $\Sigma = \{1,2,\dots,n\}$, then define $S_n := Perm(\Sigma)$ and call it the symmetric group of $n$ elements.
        \item Sometimes, it is helpful to treat a group $G$ as a subgroup of $Perm(\Sigma)$ for some set $\Sigma$ since it provides an angle to better understand G, \textit{e.g.}, $\mathrm{GL}_n(\bbC)\leq Perm(\bbC^n)$. In general, we may view G as a subgroup of $Perm(G)$ where we view $g \in G$ as $L_g \in Perm(G)~ \forall g \in G$ (or $R_g \in Perm(G)~ \forall g \in G$). If $G$ is a finite group of $n$ elements, $G$ may be viewed as a subgroup of $S_n$. This point of view will be more apparent after the introduction of the terms "monomorphism, isomorephism,...".

    \end{enumerate}
\end{example}

Let $G$ be a group and $g\in G$. Here are some definitions:
\begin{definition}
\begin{enumerate}
    \item The \emph{size of the group}, denoted \emph{$|G|$}, is called \emph{the order of the $G$}. The group $G$ is called a finite group if $|G| < \infty$; otherwise, G is an infinite group.
    \item We define $g^n$ for all cases $n\in \bbZ$: if $n=0$, $g^n =e$; if $n>0$, $g^n =gg\cdots g$ (product of n terms); if $n<0$, $g^n = g^{-1}g^{-1}\cdots g^{-1}$ (product of $-n$ terms).
    \item If there exists $n \in N$ such that $g^n = e$ , then the smallest such $n$ is called \emph{the order (or period) of $g$}, denoted \emph{$ord(g)$} or \emph{$o(g)$}. If no such n exists, then $ord(g)$ is defined to be $\infty$.
    \item \textbf{Fact.} If $G$ is finite, then $ord(g)$ is finite. (We will see later that $ord(g)$ divides $|G|$.) Proof. Since the subset $\{ g^n : n \in \bbN \} \subset G$ is finite, for some natural numbers $m>n$ we have $g^m=g^n$, which implies $g^{m-n}=e$.
    \item Let $S \subset G$ be non-empty. The subgroup of $G$ generated by $S$ is defined to be: $$
    <S>:=\left\{s_{1}^{k_{1}} s_{2}^{k_{2}} \cdots s_{n}^{k_{n}} | n \in \mathbb{N}, s_{1}, \ldots, s_{n} \in S, k_{1}, k_{2}, \ldots, k_{n} \in \mathbb{Z}\right\}
    $$ Check that this defines a subgroup!
    \item If $S$ has only one element $g$, write $<S>=<g>$ for simplicity. Check that $ord(g)=|<g>|$
\end{enumerate}
\end{definition} 

\subsection{Cosets and Lagrange's Theorem}
Let $H$ be a subgroup of a group $G$ and $g \in G$. The subset $gH := \{gh|h\in H\}\subset G$ is called a left coset (of H) with respect to $g$. The set of left cosets is denoted $G/H=\{gH|g\in G\}$ and the size $|G/H|$ is called the index of H (can be $\infty$). Of course, left cosets with respect to different g may coincide. Similarly, one can define right cosets of H and the set of right cosets is denoted $H\backslash G$.

The map $L_g: H\Rightarrow gH$ is bijective. If $G$ is finite, then $gH$ and $Hg$ are both finite sets. 

\textbf{Fact.} Every element $g\in G$ belongs to a left coset and $\forall g_1,g_2 \in G$, either $g_1 H \bigcap g_2H =\emptyset$ or $g_1 H =g_2 H$.

Proof. First $g =ge\in gH$. Second, if $g_1 H \bigcap g _2 H\neq\emptyset$, then there exists $h_1,h_2\in H$ such that $g_1h_1=g_2h_2\Rightarrow g_1=g_2h_2h_1^{-1}\in g_2H\Rightarrow g_1H\subset g_2H$ (since $H$ is closed under multiplication); similarly we have $g_2 H \subset g_1 H$.


\begin{theorem}
    Lagrange's Theorem

    If $H$ is a subgroup of a finite group $G$, then $|H|$ divides $|G|$.  
\end{theorem}

\begin{proof}
    By the above fact, $G$ is partitioned by the left cosets, that is, $G$ is a disjoint union of the left cosets: 
    $$    G=\bigcup_{g H \in G / H} g H    $$
    Then (2) above implies that $|G|=|H| \times|G / H|$.
\end{proof}

\begin{remark}
    The whole proof can be done with right cosets.

    The partition on G by left cosets actually comes from an equivalence relation: we say 
     $g_1 \sim g_2$ iff $g_1^{-1}g_2 \in H$.
\end{remark}

\begin{lemma}
    \emph{Corollaries of Lagrange's Theorem}

    Let $G$ be a finite group. 
    \begin{enumerate}
        \item If $|G|$ is a prime number, then $\{e\}$ and $G$ are the only subgroups of $G$.
        \item If $g \in G$, then $ord(g)$ divides |G|.
        \item If $g \in G$, then $g^{|G|}=e$.
        \item \emph{Fermat's little theorem}: take $G = (\bbZ/p\bbZ)^*$ and use (c)
        \item \emph{Euler's theorem}: take $G = (\bbZ/N\bbZ)^*$ and use (c)
    \end{enumerate}
\end{lemma}
	\section{Assignment}
	% \subsection{Assignment 1}

\begin{exercise}
1. Prove that $(0,1) :=\{x \in \bbR|0 < x < 1\}$ is bijective to $(0,1] :=\{x\in \bbR | 0< x\leq 1\}$.
\end{exercise}

\begin{proof}
We extract a countable set $\{x|x=1/2^n,n\in \bbN^+ \}=\{1/2,1/2^2,1/2^3,\cdots\}$ first from $(0,1)$. Map $1/2$ to $1$, that is, $1/2^0$. Map $1/2^2$ to $1/2$, while $1/2^n$ is mapped to $1/2^{n-1}$. This is the map we need. While for other domain in $(0,1)$, just map them to themselves. 
\end{proof}

\begin{exercise}
2. Let $S$ be a set. Prove that $S$ is not bijective to its power set $P(S)$.
\end{exercise}
\begin{proof}
Firstly, we consider the case of finite element number. The proposition this question asked is right for empty set $\phi$. If the element number of set $S$ is $n$, then the element number of the power set $P(S)$ is $2^n$. They never equal.

For the infinite case, \textit{Cantor's theorem}  directly makes the result.
\end{proof}


\begin{exercise}
3. Define a relation $\sim $ on $\bbN\times\bbN$ as $(a, b)\sim(c, d)$ if and only if $a+d=c+b$. Prove that $\sim$ is an equivalence relation and write down a natural bijection between the the set of equivalence classes and Z.
\end{exercise}
\begin{proof}
\textit{i.e.}, $a+d=c+b$ induces $a-b=c-d$.

\begin{itemize}
    \item If $(a,b)\in \bbN\times\bbN$, $(a,b)\sim(a,b)$ evidently.
    \item If $(a, b)\sim(c, d)$, $a+d=c+b$. Naturally, $d+a=b+c$ and we have $(c, d)\sim(a, b)$.
    \item If $(a, b)\sim(c, d)$, $(c, d)\sim(e, f)$, $a-b=c-d=e-f$. Of course, $(a, b)\sim(e, f)$.
\end{itemize}

All elements $(a,b)$ of the same equivalence class share a number $a-b\in\bbN$. Mapping the equivalence class directly to the number $a-b$ is a natural bijection.
\end{proof}

\begin{exercise}
4. Let $V$ be a non-zero vector space over $\mathbb{R}$. Prove by \textit{Zorn's lemma}  that $V$ has a basis.
\end{exercise}
\begin{proof}
We define the relation $\leq$ for the set of vectors simply by $\subset$ relation. $A=\{\text{some vectors}\} \leq B=\{\text{some vectors}\}$ as long as $A \subset B$. One may begin with a single element set $A_1=\{\vec{v_1}\}$ to try to span the whole space. If a vector in the vector space is not able to be spanned by the set $A_1$, one can always adds this vector to $A_1$ to make it $A_2=\{\vec{v_1},\vec{v_2}\}$. This compose a chain of set $A_1\leq A_2 \leq A_3 \cdots$, in which every $A_i$ is an element of partial order set $S$.  If there is not such a basis, which infers that infinite vectors are needed to represent all elements of the vector space, the vector appending procedure will never stop and the upper bound never appears. However, the appending is aborted by the limited dimension of vector space and then comes the upper bound of the chain.

With different seed $\vec{v_1}$ and different choices of appended vectors, $S$ contains a lot of chain, of which they all have upper bounds. By applying the \textit{Zorn's lemma}, there must be the maximal set of basis vectors, which is the real basis for the whole space. 
\end{proof}
	\subsection{Assignment 2}

\begin{exercise}
1. Let $G$ be a group. Prove that the order of $g$ is equal to the order of $g^{-1}$ for all $g\in G$.
\end{exercise}

\begin{proof}
We extract a countable set $\{x|x=1/2^n,n\in \bbN^+ \}=\{1/2,1/2^2,1/2^3,\cdots\}$ first from $(0,1)$. Map $1/2$ to $1$, that is, $1/2^0$. Map $1/2^2$ to $1/2$, while $1/2^n$ is mapped to $1/2^{n-1}$. This is the map we need. While for other domain in $(0,1)$, just map them to themselves. 
\end{proof}

\begin{exercise}
    Let $G$ be a group with group law $\bullet$, that is, the product of $g$ and h is $g \bullet h$ for all $g, h \in G$.
    Define a new binary operation $* : G \times G \rightarrow $G as $g * h := h \bullet g$. Prove that this defines a group structure on G (called the \textit{opposite group}).
\end{exercise}

\begin{exercise}
    Let $G$ be a group with subgroups $K$ and $H$ such that $K \leq H \leq G$. Prove the following statements.
    \begin{itemize}
        \item  The index $[G : K]$ is finite if and only if $[G : H]$ and $[H : K]$ are finite.
        \item If $[G : K]$ is finite, then $[G : K] = [G : H] \times [H : K]$.

    \end{itemize}
\end{exercise}

\begin{exercise}
    Let $G$ be a group with subgroups $K$ and $H$. Prove the following statements
    \begin{itemize}
        \item  If $K$ is normal in $G$, then $H \bigcap K$ is normal in $H$.
        \item Give an example of $K \leq H \leq G$ such that $K$ is normal in $H$ and $H$ is normal in $G$ but $K$ is not normal in $G$.
        
    \end{itemize}
\end{exercise}

\begin{exercise}
    Let $G$ be a finite subgroup of $\mathrm{GL}_n(\bbC)$ (the group of $n$ by $n$ invertible complex matrices).
Prove the following statements.
\begin{itemize}
    \item Every $g \in G$ is diagonalizable.
    \item  If $G$ is abelian, then all elements in $G$ can be simultaneously diagonalized.
\end{itemize}
\end{exercise}

\begin{exercise}
    Let $G$ be a finite group of order $pq$, where $p$ and $q$ are distinct prime numbers. Prove the
following statements.
\begin{itemize}
    \item The center $C(G)$ is either $\{e\}$ or $G$.
    \item  There exist elements $g$ and $h$ of $G$ such that $ord(g) = p$ and $ord(h) = q$ (hint:
    consider the equivalence relation on $G: g_1 \sim g_2$ if and only if $h^{-1} g_{1} h=g_{2}$ for some
    $h \in G$).
\end{itemize}
\end{exercise}





\end{document}